% !TEX encoding = UTF-8 Unicode

\chapter{Tworzenie zbiorów danych}

Wraz z postępem w tworzeniu pracy zmieniały się założenia oraz cele.
Głownymi motywatorami zmian były postępy w tworzeniu zbiorów danych oraz samych
modeli sieci. Priorytetem było stworzenie zestawu danych
umożliwiającego rozwój różnych modeli sieci neuronowych bez konieczności każdorazowego
ingerowania w zbiór lub dopasowywania go specjalnie do konkretnej sieci.\\\\
Pierwszym założeniem było zrobienie określonej ilości zdjęć kości położonej na środku
obrazu otrzymywanego z kamery, na obszarze kwadratu o boku około 3 krotnie większym
niż długość ściany samej kości. Miało to umożliwić kadrowanie obrazów
do niewielkich rozmiarów choć z kośćmi położonymi w różnych jego miejscach.\\
Kolejną kwestią była konieczność przystosowania sieci do rozpoznawania kości o
różnych kolorach ścian, oczek jak i samej barwy tła. Ponieważ docelowo sieć
miała operować na rzeczywistym obrazie z kamery, ważne było także zapewnienie poprawnego
działania w przypadku niewielkich zniekształceń obrazu bądź szumów.\\
Powyższe wymagania narzuciły więc konkretną formę zestawów danych.
Każdy zestaw ma ustaloną barwę tła, kości oraz samego oczka. Każda ścianka sfotografowana
od 3 do 30 razy, określając liczebność każdego ze zbiorów miedzy 18 a 180 zdjęć.
Zdjęcia w niektórych zestawach poddawane były działaniu zmiennego oświetlenia naturalnego
lub ostrego punktowego światła w celu wygenerowania trudniejszych do rozpoznania obrazów.
Poszczególne zestawienia kolorystyczne w tych zbiorach wypisane są w tabeli poniżej. \newpage

\begin{table}[ht]
\centering
\begin{tabular}{rcl|cc}
 \multicolumn{3}{c}{Kolor} & \multicolumn{2}{c}{Ilość zdjęć} \\ \hline
kości & oczek & tła & ściany & kostki \\ \hline
biały & czarny & czerwony & 30 & 180 \\
biały & czarny & granatowy & 3 & 18 \\
biały & czarny & czarny & 8 & 48 \\
jasne drewno & czarny & czerwony & 3 & 18 \\
czarny & biały & czarny & 10 & 60 \\
czarny & biały & czerwony & 10 & 60 \\
czerwony & biały & czerwony & 5 & 30 \\
czerwony & czarny & czerwony & 3 & 18 \\
granatowy & złoty & biały & 4 & 24 \\
granatowy & złoty & niebieski & 6 & 36 \\
zielony & biały & zielony & 8 & 48 \\
zielony & biały & biały & 5 & 30 \\
różowy & czarny & biały & 5 & 30 \\ \hline
\multicolumn{3}{c}{\textit{Łączna ilość zdjęć:}} & \textit{100} & \textit{600}
\end{tabular}
\vspace{0.2cm}
\caption{Zestawienie kolorystyczne obrazów 1}
\label{tab:zestawienie1}
\end{table}

Ilość zdjęć wykonywanych dla danych zestawień kolorystycznych była jak widać mocno
zróżnicowana. Największa ilość, zdjęcia z białą kostką, czarnymi oczkami i czerwonym
tłem, jest związana z testowaniem różnego rodzaju oświetlenia kości na wyniki uczenia
różnych modeli sieci. Przeważnie ilość zdjęć wahała się między 4 a 8 na jedną ścianę. \\
Każdy z obrazów został następnie poddany przekształceniom by zwiększyć ich ilość,
konieczną do umożliwienia prawidłowego uczenia się sieci. \\
Po zaaplikowaniu szcześciu różnych przekształceń oraz obrotom o 15\textsuperscript{o}, z każdego
obrazu uzyskano 168 zdjęć o rozmiarze 64x64 piksele. Łącznie zbiór danych składał się
ze 100800 obrazów, chociaż w niektórych przypadkach ilość zdjęć była mniejsza.
Wynikało to z analizy uczenia sieci dla obrazów z różnymi kątami obrotów poszczególnych
zdjęć. Dla tego zbioru danych ostatecznie została przyjęta wartość 15\textsuperscript{o}, jako
wartości która dawała wystarczająco liczny zbiór testowy do możliwości uczenia sieci
oraz nie utrudniała znacząco prac pod kątem ilości danych do przetransformowani przez sieć.
