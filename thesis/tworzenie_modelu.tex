% !TEX encoding = UTF-8 Unicode

\chapter{Sieć rozpoznająca ilość oczek}

W momencie zrobienia zbioru danych składających się z kwadratowych obrazów, przystąpiono
do próby stworzenia i nauczenia modelu sieci neuronowej rozpoznającego ilość oczek
wyrzuconych na kostce do gry. \\

\subparagraph{Hipotezy} \mbox{}
Przed przystąpieniem do tworzenia zbioru danych zrodziła się bardzo duża ilość pytań
oraz wątpliwości. Większość z nich dotyczyła kwestii możliwości, że sieć będzie w stanie
jakkolwiek rozpoznać ilość oczek. Obawy te wiązały się z faktem, że dla przykładu
w zbiorze MNIST, położenie cyfr na obrazie było stałe. W przypadku kiedy na obrazach
kolor biały był obszarem w kształcie przypominającym pionową kreskę, z dużym
prawdopodobieństwem można było założyć że jest to cyfra 1 lub 7, a analogiczna
sytuacja zachodziła dla innych cyfr. W rozpoznawaniu oczek na kostce, zarówno sama
kostka może być umieszczona w różnych miejscach na obszarze całego obrazu, jak również
dopuszczalny jest jej obrót o dowolny kąt. \\
Następną kwestią był niewielki rozmiar oczka w stosunku do powierzchnii całego ekranu.
W przypadku zbioru MNIST średnio około 12-15\% obrazu stanowi barwa biała, która
decyduje o wartości zwróconej przez sieć. W przypadku kwadratowych zdjęć z kostkami,
rozmiar jednego oczka to jedynie 0,21\% powierzchnii rozmiaru, a dla obrazów prostokątnych
wartość ta maleje do około 0.08\%. Oznacza to, że nawet niewielki szum lub
zniekształcenia mogą utrudnić prawidłowe rozpoznanie kości. \\

\subparagraph{Pierwszy model} \mbox{}\\
Pierwsza próba stworzenia sieci, mając na uwadze wyżej wspomniane wątpliwości, miała
na celu wytrenowanie możliwie prostego zbioru obrazów. W tym celu wykorzystano
jedynie obrazy z czerwonym tłem, białą kością o czarnymi oczkami. Dla zwiększenia
liczby zdjęć w zbiorze, zmniejszono kąt obrotu każdego z nich do 5\textsupercript{o},
uzyskując 60480 zdjęć ze 120 oryginalnych. \\
Architektura tej sieci, była dobierana bez większego wdrażania się w szczegóły i
bazowała na modelach sieci udostępnionych na stronach Keras oraz TensorFlow,
wykorzystywanych do analizy zbiorów MNIST oraz CIFAR10. Za optymalizator został wybrany
Adam a sam trening został ustalony na 25 epok w pierwszej turze oraz 10 epok w drugiej
turze. Nigdzie wcześniej nie zauważono tego typu praktyki, aby rozdzielać epoki uczenia.
Zrobiono to dlatego, aby umożliwić wcześniejsze zakończenie całego procesu w sytuacji
gdyby nie zauważono żadnych postępów. Dodatkowo pozwoliło to na zachowanie modelu
po 25 epokach, który mimo że mógłby nie osiągać dobrych rezultatów, pozwoliłby
na wyciągnięcie wniosków lub dalszą naukę. \\
Model sieci prezentował się następująco: \\\\

Rezulataty osiągnięte przez ten model były co najmniej zdumiewające. Po pierwszej epoce
sieć uzyskała 61,98\% skuteczności ostatecznie osiągając wynik 99,88\% po 25 epokach.
Kolejne 10 epok praktycznie nie zmieniło tego rezultatu. \\
Po analizie tego dokonania, okazało się że tak duża ilość zdjęć otrzymanych
z każdego ze zdjęć początkowych stworzyła zbiór w którym wiele zdjęć praktycznie się
powtarzało. Sieć nie miała żadnych problemów podczas testowania na zbiorze
do którego w teorii nie miała dostępu podczas treningu. Ten fakt spowodował konieczność
trenowania sieci na zbiorach zróżnicowanych pod kątem doboru kolorów oraz rozmiarów
kości na zdjęciach.

\subparagraph{Pierwszy model ze zróżnicowanymi zdjęciami}
Po stworzeniu modelu który udowodnił że da się zrealizować zadanie stworzenia dobrze
działającej sieci dla tego problemu, zabrano się do kolejnego etapu prac. W tym celu
wykorzystano opisany w części SŁOWNIK/ZBIORY_DANYCH zbiór złożony z 13 różnych zestawów
kolorystycznych kości, oczek oraz tła, liczący po 80640 i 20160 zdjęć na części
treningową i testową. \\
W tym modelu wykorzystano architekturę jedynie nieznacznie zmienioną w stosunku do
użytej w pierwszym modelu. Architektura ta jest widoczna poniżej: \\\\

Sieć uczona była w 25, a następnie na 10 epokach.
Po pierwszych 25 epokach uzyskano dokladność 55,64\% co potwierdziło przypuszczenie
z przedniej sieci o możliwości pokrycia się zdjęć w zbiorach treningowym i testowym.
Kolejne 10 epok poprawiło wynik sieci do 65,30\%, pozwalając przy okazji zaobserwować
pewien dość istotny fakt. W pierwszej sesji 25 epok, dokładność sieci przez ostatnie
5 epok oscylowała w okolicach 52\%. W drugiej sesji, niemalże od razu wartość
podniosła się do 56\%. Prawdopodobnie miało to związek ze sposobem w jaki dostarczane
są dane do modelu. Dane były ustalane losowo, ale dla każdej epoki w jednej sesji,
układ ten się nie zmieniał. Istniała szansa, że kiedy w następnej sesji zdjęcia
były przetwarzane w innej kolejności, umożliwiło to lepsze dopasowanie sieci i efekt
skoku jej dokładności.

\subparagraph{Model z wybranymi zdjęciami}
Znacząca różnica w dokładności między modelem z jednolitymi oraz zróżnicowanymi zdjęciami
doprowadziła do stworzenia sieci uczonej na zbiorze różnorodnych obrazów, ale
dobranymi tak, aby kontrast między tłem a kością był wyraźny. Architektura sieci
jest identyczna jak w powyższych przykładach, jedyną różnicą jest własnie przetwarzanie
wyselekcjonowanych zdjęć. \\
Zastosowanie 25 epok wystarczyło aby uzyskać dokładność na poziomie 89,23\% co jest
rezultatem zdecydowanie lepszym niż uzyskane wcześniej 55,64\%.

\subparagraph{Analiza różnych optymalizatorów}
Jeśli sama różnica w doborze konkretnych obrazów potrafi tak bardzo zmienić wyniki
otrzymane przez sieć, podjęto próbę przetestowania kilku z optymalizatorów dla
identycznych sieci oraz zbiorów danych. W tym celu postanowiono użyć optymalizatorów
RMSprop oraz SGD, opisanych dokładniej w sekcji SŁOWNIK/OPTYMALIZATORY.\\

Oba modele z optymalizatorami RMSprop oraz SGD, zostały podobnie jak wcześniej opisany
model z optymalizatorem Adam poddane uczeniu przez 25 epok różnorodnych zbiorów obrazów.
Wynik dla RMSprop był zdecydowanie najlepszy i wyniósł 84,34\% skuteczności.
Najgorszy w zestawieniu model korzystający z SGD uzyskał jedynie 36,92\% poprawności.
Pośrodku obu, okazał się Adam który jak opisane jest wyżej, uzyskał 55,64\% po sesji
uczenia przez 25 epok.\\
Wyniki te dowiodły, że optymalizator jesty kluczowym parametrem \textit{(ang. hyperparameter)}
dla odpowiednio skutecznego uczenia. Uzyskany wynik dla RMSprop jest sporym zaskoczeniem,
ponieważ w licznych tekstach naukowych to Adam uznawany jest za jeden z najlepszych
optymalizatorów przez co cieszy się ogromną popularnością. Również z tego powodu, pomimo
gorszego niż RMSprop wyniku, Adam będzie wykorzystywany w praktycznie wszystkich następnych
modelach.

\paragraph{Wypis stworzonych modeli sieci}

dice_cnn 64x64x1 pierwszy model ever, 99.88\% accuracy, 25 i 10 epok (prawie poprawy) \\\\

dice_cnn_continue 64x64x1 pierwszy model,tylko douczanie \"distance\" 51\% accuracy, 10 epok \\\\

dice_cnn_adam 64x64x1 pierwszy model z różnymi zdjęciami, 55\% accuracy, w continue 65\%, 25 epok, 2048 batch\\\\

dice_cnn_adam_wybrane 64x64x1 architektura jak dice_cnn_adam, zdjecia jedynie te z dużym
kontrastem (np bez blackonblack), skutecznosc 89\% 25 epok, 2048 batch \\\\

dice_cnn_rmsprop 64x64x1 architektura jak dice_cnn_adam, RMSprop zamiast Adam, 84\%, po
douczeniu 10epok 90\% 25 epok, 2048 batch \\\\

dice_cnn_sgd 64x64x1 architektura prawie jak dice_cnn_adam, 256 zamiast 196 w 2-giej Dense,
37\%, 25 epok, 2048 batch \\\\
------------------------------------------\\\\
dice_cnn_API 64x64x1 architektura docelowo jak dice_cnn_adam, pierwszy model API i pierwsze
odkrycie roznic API vs Sequential, 95\%, 25 epok, 1024 batch \\\\

dice_AlexNet 64x64x1, głęboka sieć oparta na AlexNet z ILSVRC,  98.88\% 25 epok, 2048 batch \\\\

generator_comparison 64x64x3 porównanie uczenia 3 kanałów RGB z BW, 98.4\% 20 epok, 64 batch, 18s na epoke \\\\

generator_comparison_gray 64x64x1 porównanie, większy batchsize dla BW niż RGB, 98.1\% 20 epok, 256 batch, 17s na epoke \\\\

API_Sequential_®Keras  640x480x1 różnica między API a Sequential, błedy kompilacji \\\\

API_Sequential_®Keras2 120x120x1 różnica miedzy API a Sequential, poprawny \\\\

generator_320x240_AlexNet, 320x240x3 !!, nienauczona, 20 epok, 64 batch \\\\

generator2_320x240_AlexNet, 320x240x1, nienauczona, 20 epok, 64 batch \\\\

generator2_320x240, 320x240x1, nienauczona, 20 epok, 64 batch \\\\

simple_NN_106x79 106x79x1, pierwsza nauczona z prostokatnymi obrazami, filtry konwolucyjne
nie sa tutaj kwadratowe, 68\%, 20 epok, 512 batch \\\\

simple_NN_160x120 160x120x1, nienauczona, 20 epok, 256 batch \\\\

substituting_106x79 106x79x1, próba przyśpieszenia pracy przez redukcje obliczen w
simple_106x79 jednak zastosowano kwadratowe filtry, byc moze przez to - nie nauczona, 20 epok, 256 batch \\\\

substituting_160x120 160x120x1, próba przyśpieszenia pracy przez redukcje obliczen w
simple_106x79 jednak zastosowano kwadratowe filtry, byc moze przez to - nie nauczona, 20 epok, 256 batch \\\\

subst_LReLU_106x79 106x79x1, próba przyśpieszenia przez redukcje ilosci obliczen wraz
z zastosowaniem LeakyReLU zamiast ReLU, nienauczona, 20 epok, 1024 batch \\\\
