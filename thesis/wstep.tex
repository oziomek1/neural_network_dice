% !TEX encoding = UTF-8 Unicode

\chapter{Wstęp}

\section{Biologiczny neuron}

Neuron to komórka nerwowa zdolna do przewodzenia i przetwarzania sygnału
elektrycznego, w którym zawarta jest informacja. Jest on podstawowym elementem
układu nerwowego wszystkich zwierząt. Każdy neuron składa się z
ciała komórki (soma, neurocyt) otaczającego jądro komórkowe, neurytu (akson)
odpowiedzialnego za przekazywanie informacji z ciała komórki do kolejnych neuronów
oraz dendrytów służących do odbierania sygnałów i przesyłaniu ich do ciała
komórkowego \textit{Budowa neuronu} \cite{CS231n, NNbiology}.
Impuls elektryczny między neuronami przekazywany jest
w synapsie, miejscu komunikacji neuronu z następnym. Synapsa składa się
z części presynaptycznej (aksonu) i postsynaptycznej (dendrytu). Neuron
przewodzi sygnał tylko w sytuacji, kiedy suma potencjałów wejściowych od innych
neuronów na jego dendrytach przekroczy określony poziom. W przeciwnym wypadku
neuron nie przewodzi sygnału. Co więcej, zwiększenie potencjału na wejściach
nie powoduje wzmocnienia potencjału na wyjściu neuronu.\\
Neurony połączone i działające w ten sposób tworzą sieci neuronowe. Najdoskonalszą
znaną siecią neuronową jest mózg człowieka. Przeciętnie posiada on około 100
miliardów neuronów, każdy z nich połączony jest z około 10 tysiącami innych
neuronów przez połączenia synaptyczne. Liczba połączeń synaptycznych szacowana
jest na około 10\textsuperscript{15}.

\section{Sztuczny neuron}

Matematycznym modelem neuronu jest tzw. neuron McCullocha-Pittsa, nazywany
również neuronem binarnym. Jest to prosta koncepcja zakładająca, że każdy neuron
posiada wiele wejść, z których każde ma przypisaną wagę w postaci liczby
rzeczywistej oraz jedno wyjście. Wyjściem neuronu jest wartość funkcji aktywacji
dla argumentu będącego sumą wszystkich wag pomnożonych przez odpowiednie
wartości wejściowe. Wyjście danego neuronu połączone jest z wejściami innych
neuronów, tak jak ma to miejsce w przypadku biologicznego neuronu \textit{Sztuczny neuron} \cite{CS231n}.\\
Tak zbudowany, sztuczny model neuronu jest podstawowym składnikiem sztucznych sieci
neuronowych z racji prostoty działania i łatwości implementacji.\\
Wykorzystanie pojedynczego neuronu nie pozwala na rozwiązywanie skomplikowanych zadań.
Przykładowo próba realizacji prostych funkcji logicznych ANR, OR, NOT i XOR okazuje
się możliwa jedynie dla trzech pierwszych podanych funkcji \textit{Rozwinięcie zagadnienia} \cite{XORproblem}.
Problem wiąże się z brakiem możliwości uzyskania poprawnych rezultatów dla zbiorów które
nie są liniowo separowalne, czego przykładem jest właśnie funkcja XOR. W takich przypadkach
konieczne jest użycie większej ilości neuronów. Tworzenie rozbudowanych struktur z pojedynczych
neuronów określanych mianem sieci neuronowych, znacząco zwiększa możliwości adaptacyjne
dla poszczególnych problemów, niejednokrotnie zaskakując osiąganymi wynikami.

\section{Historia}

Rozpoczęcie prac nad sztucznymi neuronami można datować na rok 1943 kiedy to
Warren McCulloch i Walter Pitts przedstawili wspomniany już wcześniej model \textit{Historia} \cite{NNbiology}.
Pierwsze sieci neuronowe używane były do operacji bitowych i przewidywania kolejnych
wystąpień bitów w ciągu. Mimo licznych prób zastosowania ich do realnych problemów
nie zyskały one popularności. Powodem tego były prace naukowe sugerujące ograniczenia
sieci takie jak brak możliwości rozszerzenia sieci do więcej niż jednej warstwy oraz
jedynie jednokierunkowe połączenie między neuronami. W początkowym okresie rozwoju
sieci neuronowych przyjmowano także wiele błędnych założeń, które wraz z ograniczonymi
możliwościami obliczeniowymi komputerów skutecznie zniechęcały naukowców do prac
nad tym zagadnieniem.\\
Przełomem okazał się rok 1982, kiedy to John Hopfield przedstawił sieć asocjacyjną
(zwaną siecią Hopfielda). Nowością było dwukierunkowe połączenie neuronów, co
zapewniało możliwość uczenia się danych wzorców. Kolejnymi przełomowymi odkryciami
były zarówno wprowadzenie wielowarstwowych sieci neuronowych oraz wstecznej
propagacji. Nowe odkrycia pozwoliły na zastosowanie sieci w wielu różnych
dziedzinach, wymagając jednak dużych ilości obliczeń, obszernych zbiorów
treningowych, wielu tysięcy iteracji i długiego czasu potrzebnego na wytrenowanie.
Korzyścią płynącą z zastosowania sieci neuronowej jest fakt, że wytrenowany model
dzięki zapisanym wartościom wag neuronów można wykorzystać natychmiastowo bez konieczności
uczenia.
