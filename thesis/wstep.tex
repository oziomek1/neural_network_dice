% !TEX encoding = UTF-8 Unicode

\chapter{Wstęp}

Sieci neuronowe składają się ze sztucznych neuronów, które są uproszczonymi modelami ich
biologicznych odpowiedników, a pierwsze prace nad nimi zaczęły sie już w latach 50.
XX wieku.

\section{Sztuczny neuron}

Podstawowym modelem neuronu jest tzw. neuron McCullocha-Pittsa, nazywany
również neuronem binarnym. Jest to prosta koncepcja zakładająca, że każdy neuron
posiada wiele wejść, z których każde ma przypisaną wagę w postaci liczby
rzeczywistej oraz jedno wyjście. Wyjściem neuronu jest wartość funkcji
dla argumentu będącego sumą wszystkich wag pomnożonych przez odpowiednie
wartości wejściowe, określana jako funkcja aktywacji, wyjaśniona w późniejszych rozdziałach.
Wyjście danego neuronu połączone jest z wejściami innych
neuronów, tak jak ma to miejsce w przypadku biologicznego neuronu \cite{CS231n}.\\
Tak zbudowany, sztuczny model neuronu jest podstawowym składnikiem sztucznych sieci
neuronowych z racji prostoty działania i łatwości implementacji.\\
Wykorzystanie pojedynczego neuronu nie pozwala na rozwiązywanie skomplikowanych zadań.
Przykładowo próba realizacji prostych funkcji logicznych ANR, OR, NOT i XOR okazuje
się możliwa jedynie dla trzech pierwszych podanych funkcji \cite{XORproblem}.
Problem wiąże się z brakiem możliwości uzyskania poprawnych rezultatów dla zbiorów które
nie są liniowo separowalne, czego przykładem jest właśnie funkcja XOR. W takich przypadkach
konieczne jest użycie większej ilości neuronów. Tworzenie rozbudowanych struktur z pojedynczych
neuronów określanych mianem sieci neuronowych, znacząco zwiększa możliwości adaptacyjne
dla poszczególnych problemów, niejednokrotnie zaskakując osiąganymi wynikami.

\section{Historia}

Rozpoczęcie prac nad sztucznymi neuronami można datować na rok 1943 kiedy to
Warren McCulloch i Walter Pitts przedstawili wspomniany już wcześniej model \cite{NNbiology}.
Pierwsze sieci neuronowe używane były do operacji bitowych oraz przewidywania kolejnych
wystąpień bitów w ciągu. Mimo licznych prób zastosowania ich do realnych problemów
nie zyskały one popularności. W początkowym okresie rozwoju sieci neuronowych
przyjmowano także wiele błędnych założeń, które wraz z ograniczonymi możliwościami
obliczeniowymi komputerów skutecznie zniechęcały naukowców do prac nad tym zagadnieniem.\\
