% !TEX encoding = UTF-8 Unicode

\chapter {Wstęp}

\section {Biologiczny neuron}

Neuron to komórka nerwowa zdolna do przewodzenia i przetwarzania sygnału elektrycznego w którym zawarta jest informacja. Jest on podstawowym elementem układu nerwowego wszystkich zwierząt. Każdy neuron składa się z ciała komórki (soma, neurocyt) otaczającego jądro komórkowe, neurytu (akson) odpowiedzialny za przekazywanie informacji z ciała komórki do kolejnych neuronów oraz dendrytów służących do odbierania sygnałów i przesyłaniu ich do ciała komórkowego. Impuls elektryczny z jednego neuronu do drugiego przekazywany jest w synapsie, miejscu komunikacji danego neuronu z poprzednim. Synapsa składa się z części presynaptycznej (aksonu) i postsynaptycznej (dendrytu). Neuron przewodzi sygnał tylko w sytuacji kiedy suma potencjałów na wejściach od innych neuronów na jego dendrytach przekroczy określony poziom. W przeciwnym wypadku neuron nie przewodzi sygnału. Dodatkowo, zwiększenie potencjału na wejściach nie powoduje wzmocnienia potencjału na wyjściu neuronu. \\
Neurony połączone i działające w ten sposób tworzą sieci neuronowe, którym dobrym przykładem może być mózg człowieka.  Przeciętnie posiada on około 100 miliardów neuronów, każdy z nich połączony jest z około 10 tysiącami innych neuronów przez połączenia synaptyczne. Liczba połączeń synaptycznych szacowana jest na około 10\textsuperscript{15}.

\section {Sztuczny neuron}

\section {Historia}