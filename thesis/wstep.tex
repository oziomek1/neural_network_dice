% !TEX encoding = UTF-8 Unicode

\chapter {Wstęp}

\section {Biologiczny neuron}

Neuron to komórka nerwowa zdolna do przewodzenia i przetwarzania sygnału
elektrycznego w którym zawarta jest informacja. Jest on podstawowym elementem
układu nerwowego wszystkich zwierząt. Każdy neuron składa się z
ciała komórki (soma, neurocyt) otaczającego jądro komórkowe, neurytu (akson)
odpowiedzialny za przekazywanie informacji z ciała komórki do kolejnych neuronów
oraz dendrytów służących do odbierania sygnałów i przesyłaniu ich do ciała
komórkowego. Impuls elektryczny z jednego neuronu do drugiego przekazywany jest
w synapsie, miejscu komunikacji danego neuronu z poprzednim. Synapsa składa się
z części presynaptycznej (aksonu) i postsynaptycznej (dendrytu). Neuron
przewodzi sygnał tylko w sytuacji kiedy suma potencjałów na wejściach od innych
neuronów na jego dendrytach przekroczy określony poziom. W przeciwnym wypadku
neuron nie przewodzi sygnału. Dodatkowo, zwiększenie potencjału na wejściach
nie powoduje wzmocnienia potencjału na wyjściu neuronu. \\
Neurony połączone i działające w ten sposób tworzą sieci neuronowe, których
dobrym przykładem może być mózg człowieka. Przeciętnie posiada on około 100
miliardów neuronów, każdy z nich połączony jest z około 10 tysiącami innych
neuronów przez połączenia synaptyczne. Liczba połączeń synaptycznych szacowana
jest na około 10\textsuperscript{15}.

\section {Sztuczny neuron}

Matematycznym modelem neuronu jest tzw. neuron McCullocha-Pittsa, nazywany
również neuronem binarnym. Jest to prosta koncepcja zakładająca, że każdy neuron
posiada wiele wejść, z których każde ma przypisaną wagę w postaci liczby
rzeczywistej oraz jedno wyjście. Wyjściem neuronu jest wartość funkcji aktywacji
dla argumentu którym jest suma wszystkich wag pomnożonych przez odpowiednie
wartości wejściowe. Wyjście danego neuronu połączone jest z wejściami innych
neuronów, tak jak ma to miejsce w przypadku biologicznego neuronu.
Powyższy, sztuczny model neuronu jest podstawowym budulcem sztucznych sieci
neuronowych z racji prostoty działania i łatwości implementacji.
W rzeczywistych zastosowaniach używane są pojecia takie jak wektor wejściowy
oraz wektor wag, które oznaczają odpowiednio wszystkie wartości wejścioweoraz
wszystkie wagi dla danego neuronu.
Pojedynczyn neuron pozwala na uzyskanie mocno ograniczonych rozwizań. Przykładowo
korzystając z prostych funkcji logicznych ANR, OR, NOT, XOR okazuje się że
pojedynczy neuron jest w stanie poprawnie rozwiązywać jedynie pierwsze trzy z
podanych funkcji. Problem wiąże się z brakiem możliwości uzyskania poprawnych
rezultatów dla zbiorów które nie są liniowo separowalne. W takich przypadkach
konieczne jest użycie większej ilości neuronów, a więc tworzenie sieci neuronowych
których możliwości adaptacyjnego do poszczególnych problemów są niejednokrotnie
bardzo zaskakujące.


\section {Historia}
