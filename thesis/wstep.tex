% !TEX encoding = UTF-8 Unicode

\chapter{Wstęp}

Sieci neuronowe składają się ze sztucznych neuronów, które są uproszczonymi modelami ich
biologicznych odpowiedników. Pierwsze prace nad tym zagadnieniem zaczęły sie już w latach 50.
XX wieku.

\section{Sztuczny neuron}

Najprostszym modelem sztucznego neuronu jest tzw. neuron McCullocha-Pittsa, nazywany
również neuronem binarnym. Taki neuron posiada wiele wejść z których każde ma przypisaną
wagę w postaci liczby rzeczywistej. Suma wartości wejściowych mnożona jest przez odpowiadające
im wagi i podawana jako argument do funkcji aktywacji, wyjaśnionej w późniejszych rozdziałach.
Wartość tej funkcji jest wyjściem neuronu i staje się wejściem innych neuronów,
tak jak ma to miejsce w przypadku biologicznego neuronu \cite{CS231n}.
Tak zbudowany, sztuczny model neuronu jest podstawowym składnikiem sztucznych sieci
neuronowych.\\
Neurony najczęściej łączone są w struktury nazywane sieciami neuronowymi, ponieważ wykorzystanie
pojedynczego neuronu nie pozwala na rozwiązywanie skomplikowanych zadań.
Przykładowo próba realizacji prostych funkcji logicznych ANR, OR, NOT i XOR okazuje
się możliwa jedynie dla trzech pierwszych podanych funkcji \cite{XORproblem}.
Problem wiąże się z brakiem możliwości uzyskania poprawnych rezultatów dla zbiorów które
nie są liniowo separowalne, czego przykładem jest właśnie funkcja XOR. Tworzenie rozbudowanych
struktur z pojedynczych neuronów znacząco zwiększa możliwości adaptacyjne
dla poszczególnych problemów, niejednokrotnie zaskakując osiąganymi wynikami.

\section{Historia}

Rozpoczęcie prac nad wspomnianym wyżej sztucznym neuronem można datować na rok 1943 kiedy
Warren McCulloch i Walter Pitts przedstawili pierwszy jego model. \cite{NNbiology}.
Pierwszymi zastosowaniami sieci neuronowych były operacji bitowe oraz przewidywania kolejnych
wystąpień bitów w ciągu. Mimo licznych prób zastosowania ich do realnych problemów
nie zyskały one popularności. W początkowym okresie rozwoju sieci neuronowych
przyjmowano także wiele błędnych założeń, które wraz z ograniczonymi możliwościami
obliczeniowymi komputerów skutecznie zniechęcały naukowców do prac nad tym zagadnieniem.\\
