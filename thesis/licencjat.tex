% !TEX encoding = UTF-8 Unicode

\documentclass[a4paper,oneside,12pt]{report}

\usepackage{titlesec}
\usepackage[top=2cm, bottom=2cm, left=2cm, right=2cm]{geometry}
\usepackage[T1]{fontenc}
\usepackage[polish]{babel}
\usepackage[utf8]{inputenc}
\linespread{1.5}
\usepackage{textgreek}
\usepackage{lmodern}
\usepackage{amssymb}
\usepackage{amsmath}
\usepackage{fixltx2e}

\usepackage{setspace}
\selectlanguage{polish}
\makeatletter

\renewcommand{\maketitle}{\begin{titlepage}

	\vspace*{ \stretch{1} }
	\begin{center} \LARGE
	Uniwersytet Jagielloński \\
  	Wydział Fizyki, Astronomii i Informatyki Stosowanej
	\end{center}

	\vspace{ \stretch{1} }
	\begin{center} \Huge
		\textsc {\@title}
	\end{center}

	\vspace{ \stretch{1} }
	\begin{center}
		\large \@author \\
		\vspace{3mm}
		\normalsize Nr albumu: 1124802
	\end{center}

	\vspace{ \stretch{1.5} }
	\begin{flushright}
	\begin{minipage}{8cm}
	\begin{center} \large
	Praca wykonana pod kierunkiem \\
	prof. dr hab. Piotra Białasa \\
	kierownika Zakładu Technologii Gier \\
	FAIS UJ
	\end{center}
	\end{minipage}
	\end{flushright}
	\vspace{ \stretch{1} }
	\begin{center}
	\@date
	\end{center}
\end{titlepage}
}
\makeatother

\titleformat{\chapter}[display]
  {\normalfont\bfseries}{}{0pt}{\Huge}

\renewcommand*\contentsname{Spis Treści}

\author{Wojciech Ozimek}
\title{Wykorzystanie sztucznych sieci neuronowych do analizy obrazów na przykładzie kostki do gry}
\date{kwiecień 2018}

\begin{document}
\maketitle{}

\tableofcontents
\newpage


% !TEX encoding = UTF-8 Unicode

\chapter{Wstęp}

\section{Biologiczny neuron}

Neuron to komórka nerwowa zdolna do przewodzenia i przetwarzania sygnału
elektrycznego w którym zawarta jest informacja. Jest on podstawowym elementem
układu nerwowego wszystkich zwierząt. Każdy neuron składa się z
ciała komórki (soma, neurocyt) otaczającego jądro komórkowe, neurytu (akson)
odpowiedzialny za przekazywanie informacji z ciała komórki do kolejnych neuronów
oraz dendrytów służących do odbierania sygnałów i przesyłaniu ich do ciała
komórkowego. Impuls elektryczny z jednego neuronu do drugiego przekazywany jest
w synapsie, miejscu komunikacji danego neuronu z poprzednim. Synapsa składa się
z części presynaptycznej (aksonu) i postsynaptycznej (dendrytu). Neuron
przewodzi sygnał tylko w sytuacji kiedy suma potencjałów na wejściach od innych
neuronów na jego dendrytach przekroczy określony poziom. W przeciwnym wypadku
neuron nie przewodzi sygnału. Dodatkowo, zwiększenie potencjału na wejściach
nie powoduje wzmocnienia potencjału na wyjściu neuronu. \\
Neurony połączone i działające w ten sposób tworzą sieci neuronowe, których
dobrym przykładem może być mózg człowieka. Przeciętnie posiada on około 100
miliardów neuronów, każdy z nich połączony jest z około 10 tysiącami innych
neuronów przez połączenia synaptyczne. Liczba połączeń synaptycznych szacowana
jest na około 10\textsuperscript{15}.

\section{Sztuczny neuron}

Matematycznym modelem neuronu jest tzw. neuron McCullocha-Pittsa, nazywany
również neuronem binarnym. Jest to prosta koncepcja zakładająca, że każdy neuron
posiada wiele wejść, z których każde ma przypisaną wagę w postaci liczby
rzeczywistej oraz jedno wyjście. Wyjściem neuronu jest wartość funkcji aktywacji
dla argumentu którym jest suma wszystkich wag pomnożonych przez odpowiednie
wartości wejściowe. Wyjście danego neuronu połączone jest z wejściami innych
neuronów, tak jak ma to miejsce w przypadku biologicznego neuronu. \\
Powyższy, sztuczny model neuronu jest podstawowym budulcem sztucznych sieci
neuronowych z racji prostoty działania i łatwości implementacji.
W rzeczywistych zastosowaniach używane są pojecia takie jak wektor wejściowy
oraz wektor wag, które oznaczają odpowiednio wszystkie wartości wejścioweoraz
wszystkie wagi dla danego neuronu. \\
Pojedynczyn neuron pozwala na uzyskanie mocno ograniczonych rozwizań. Przykładowo
korzystając z prostych funkcji logicznych ANR, OR, NOT, XOR okazuje się że
pojedynczy neuron jest w stanie poprawnie rozwiązywać jedynie pierwsze trzy z
podanych funkcji. Problem wiąże się z brakiem możliwości uzyskania poprawnych
rezultatów dla zbiorów które nie są liniowo separowalne. W takich przypadkach
konieczne jest użycie większej ilości neuronów, a więc tworzenie sieci neuronowych
których możliwości adaptacyjnego do poszczególnych problemów są niejednokrotnie
bardzo zaskakujące.


\section{Historia}

Rozpoczęciem prac nad sztucznymi neuronami można datować na rok 1943 kiedy to
Warren McCulloch i Walt er Pitts przedstawili wspomniany już wcześniej model
neuronu. Pierwsze sieci neurone używane były np do operacji bitowych i
przewidywania kolejnych wystąpień bitów w ciągu. Mimo licznych prób zastosowania
ich do realnych problemów nie zyskały one popularności. Powodem tego były prace
naukowe sugerujące ograniczenia sieci takie jak brak możliwości rozszerzenia sieci
do więcej niż jednej warstwy oraz jedynie jednokierunkowe połączenie między
neuronami. W początkowym okresie rozwoju sieci neuronowych przyjmowano także wiele
błędnych założeń które wraz z ograniczonymi możliwościami obliczeniowymi komputerów
skutecznie zniechęcały naukowców do prac nad tym zagadnieniem. \\
Przełomem okazał sie rok 1982 kiedy to John Hopfield przedstawił sieć asosjacyjną
(zwaną siecią Hopfielda). Nowością było dwukierunkowe połączenie neuronów co
zapewniało możliwość uczenia się danych wzorców. Kolejnymi przełomowymi odkryciami
były zarówno wprowadzenie wielowarstwowych sieci neuronowych oraz wstecznej
propagacji. Nowe odkrycia pozwoliły na zastosowanie sieci w wielu różnych
dziedzinach, wymagając jednak dużych ilości obliczeń, obszernych zbiorów
treningowych, wielu tysięcy iteracji i długiego czasu nauki. Korzyścią jest jednak
fakt, że wytrenowany model można wykorzystać w praktycznie każdych warunkach
i natychmiastowo bez konieczności uczenia.


% !TEX encoding = UTF-8 Unicode

\chapter{Cel pracy}

Celem pracy jest zastosowanie sieci neuronowych do rozpoznawania obrazów.
Zadanie polega na rozpoznaniu ilość oczek wyrzuconych na kostce do gry.
Sieć powinna rozpoznawać kostki o dowolnym zestawieniu kolorystycznym
ścian i oczek oraz działać na rzeczywistym obrazie przesyłanym z kamery.\\
Dodatkowym aspektem poruszonym w pracy jest porównanie modeli w zależności od
architektury sieci oraz zbiorów danych.


% !TEX encoding = UTF-8 Unicode

\chapter{Technologie}
\section{Język programowania i środowisko}

\paragraph{Python} \mbox{}\\
Język programowania Python obecnie jest bardzo popularnym narzędziem wykorzystywanym
w pracy naukowej. Jest to spowodowane bardzo czytelną i zwięzłą składnią,
która w pełni pozwala skupić się na danym problemie. Python jest on także domyślnym
językiem używany w bibliotekach wykorzystywanych do tworzenia modeli sieci neuronowych.

\paragraph{Jupyter Notebook} \mbox{}\\
Aplikacja Jupyter Notebook pozwala uruchamiać w przeglądarce pliki, nazywane notebookami,
które składają się z wielu bloków. W blokach może znajdować się wykonywalny kod programu
lub jego fragment, a w pozostałych można prezentować m.in. teksty, wykresy bądź tabele,
co zdecydowanie ułatwia pracę.\\
W tej pracy wykorzystywana jest to uruchamiania programów w języku Python.

\section{Biblioteki}

\paragraph{OpenCV} \mbox{}\\
Biblioteka funkcji do obróbki obrazów, najczęsciej wykorzystywana w językach C++
oraz Python. W projekcie została użyta do uzyskiwania zbiorów obrazów kości do gry
ze zdjęć wykonanych kamerą.

\paragraph{TensorFlow} \mbox{}\\
Biblioteka do uczenia maszynowego oraz tworzenia sieci neuronowych. Jej ogromną
zaletą jest implementacja w języku C++ umożliwiająca wykorzystywanie procesorów oraz
kart graficznych. Z uwagi na charakter wykonywanych obliczeń praca przy użyciu
kart graficznych jest kilkukrotnie szybsza niż na procesorze, co znacząco przyśpiesza
proces uczenia. Biblioteka najlepiej wspiera języki programowania do C++ oraz Python.

\paragraph{Keras} \mbox{}\\
Biblioteka do tworzenia modeli sieci neuronowych, wykorzystująca bardziej niskopoziomowe
biblioteki jak TensorFlow, Theano lub CNTK. Zaletami Kerasa są
zarówno przystępny interfejs pozwalający w krótkim czasie stworzyć model sieci
oraz możliwość tworzenia zaawansowanych modeli przy średnim pogorszeniu czasu
uczenia się sieci o 3-4\% w stosunku do bibliotek, na których bazuje.\\
W sytuacji kiedy interfejs oraz dokumentacja do TensorFlow mogą być dla początkującej
osoby niezrozumiałe, jest to świetna alternatywa do wdrożenia się w to zagadnienie.

\paragraph{Numpy} \mbox{}\\
Moduł języka Python umożliwiający wykonywanie zaawansowanych operacji na macierzach
oraz wektorach, wspierający liczne funkcje matematyczne. Jest bardzo rozpowszechniony
i wykorzystywany w wielu, głównie naukowych zastosowaniach. Numpy wprowadza własne
typy danych oraz funkcje niedostępne w standardowej instalacji Pythona.

\paragraph{Matplotlib} \mbox{}\\
Narzędzie do tworzenia wykresów dla języka Python oraz modułu Numpy. Z uwagi na
bardzo duże możliwości jest bardzo popularny, pozostając jednocześnie prostym w
użyciu. Zawiera moduł pyplot, który w założeniu ma maksymalnie przypominać interfejs
w programie MATLAB.

\paragraph{LaTeX} \mbox{}\\
Oprogramowanie do składu tekstu, nie będące edytorem tekstowy typu WYSIWYG.
LaTeX bazuje na TeX, który jest systemem składu drukarskiego do prezentacji w
formie graficznej. Ogromną zaletą jest możliwość tworzenia w tekście zaawansowanych
wzorów matematycznych. Tekst niniejszej pracy napisany został przy pomocy tego narzędzia.

\section{Technologie wspomagające}

\paragraph{NVIDIA CUDA} \mbox{}\\
Równoległa architektura obliczeniowa firmy NVIDIA pozwala na wielokrotne
przyśpieszneie obliczeń podczas uczenia się sieci neuronowych. Dzięki bibliotekom
takim jak TensorFlow lub Keras, które wspierają obliczenia na kartach graficznych
czas precesu uczenia drastycznie maleje. Podczas tej pracy wykorzystana została
karta NVIDIA TESLA K80 12GB GPU dostępna na Amazon AWS oraz Google Compute Engine.

\paragraph{Amazon AWS EC2} \mbox{}\\
Platforma z wirtualnymi maszynami zwanymi instancjami, które można dostosować
zależnie od potrzeb klienta. W tej pracy zostały wykorzystane instancje
zoptymalizowane do obliczeń na kartach graficznych i uczenia maszynowego.
Dzięki jej wykorzystaniu, czas uczenia sieci zmniejszył się średnio 6-10 krotnie
w stosunku do pracy na komputerze wykorzystującym procesor.


!TEX encoding = UTF-8 Unicode

\chapter{Słownik pojęć}
\section{Zbiór danych}
Zbiór danych, obrazów lub zestaw danych to określenie wszystkich zdjęć kości wykonanych na
potrzeby pracy. Każde zdjęcie w zbiorze jest przetworzone w celu zmniejszenia czasu uczenia
się sieci oraz potrzebnej pamięci. Zdjęcia były wykonywane kamerą o rozdzielczości 1600x1200
pikseli. Każde ze zdjeć w zbiorze zostało poddane procesowi skalowania oraz kadrowania celem
osiągnięcia żądanego rozmiaru. Chcąc uzyskać większą liczebności zbioru, wszystkie obrazy
zostały dodatkowo poddane operacji obrotu o dany kąt. Każdy ze zbiorów został zduplikowany
i poddany konwersji z trybu RGB na skalę szarości przez usunięcie informacji o barwie
oraz nasyceniu kolorów, pozostawiając jedynie informację o jasności piksela.\\
Po przeprowadzeniu całego procesu, każdy obraz miał wymiary 64x64, zarówno w wersji
kolorowej i czarno białej, co skutkowało rzeczywistymi rozmiarami odpowiednio 64x64x3
oraz 64x64x1, gdzie ostatnia cyfra informuje o ilości kanałów.\\
Każdy element w zbiorze ma przypisaną wartość liczbową informującą o faktycznej ilości
oczek wyrzuconych na kostce przedstawianej na zdjęciu. Wartość ta zwana jest także
odpowiedzią i jest wykorzystywana w procesie uczenia sieci jako docelowa informacja,
którą ma zwrócić sieć po weryfikacji danego obrazu.

\paragraph{Zbiór treningowy} \mbox{}\\
Część zbioru danych wykorzystywana w procesie uczenia sieci określana jest mianem zbioru
treningowego lub zbioru uczącego. Jego liczebność to zazwyczaj 60-80\% całego zbioru danych.
Praktycznie we wszystkich zastosowaniach dane w tym zbiorze przed rozpoczęciem uczenia
poddawane są losowej permutacji.

\paragraph{Zbiór testowy} \mbox{}\\
Zbiór testowy lub zbiór walidacyjny służy do oceny zdolności wytrenowanej sieci do
rozpoznawania danych. Celem rozdzielenia tego zbioru od danych testowych jest weryfikacja
sieci na danych, które wcześniej nie zostały przetworzone przez sieć.

\section{Budowa sieci neuronowej}
\paragraph{Sieć neuronowa} \mbox{}\\
Sieć neuronowa bądź sztuczna sieć neuronowa \textit{(ang. ANN Artificial Neural Network)}
jest strukturą matematyczną, która powinna odzwierciedlać uproszczone działanie biologicznych
sieci neuronowych \textit{Sieć neuronowa} \cite{intuitiveExplanation, WIKIcnn}. Ma możliwość uczenia się poprzez obliczenia i przetwarzanie sygnałów
w elementach określanych neuronami. W dziedzinie rozpoznawaniu obrazów są w stanie zidentyfikować
elementy na obrazach, bez wcześniejszej znajomości lub wiedzy na temat przedmiotu podlegającego
rozpoznaniu. Realizują to poprzez analizę przykładowych obrazów z informacją czy znajduje się
na nich pożądany obiekt, a następnie zmianę swoich własnych parametrów w celu poprawnej
identyfikacji kolejnych zdjęć. Ten rodzaj uczenia się określany jest mianem uczenia nadzorowanego,
gdzie każdy obraz ma przypisaną wartość.\\
Zastosowanie sieci jest bardzo szerokie i wykraczające poza rozpoznawanie obrazów, jednak
w związku z powiązaniem tego zagadnienia z tematyką pracy, podany powyżej przykład ma
jak najlepiej oddać istotę działania sieci neuronowych.

\paragraph{Neuron} \mbox{}\\
Najmniejszy element sieci neuronowej, posiada wiele wejść i jedno wyjście. Może
zawierać także próg (ang. threshold), mogący ulec zmianie przez funkcje uczącą \cite{CS231n_activ, NNbiology, NeuronAnimation}.
Neuron wyposażony jest także w funkcję aktywacji, odpowiednio modyfikującą jego wyjście.
W sieciach neuronowych wyjście każdego neuronu połączone jest z wejściami neuronów
w warstwie następnej.
\begin{equation}
f(x_{i}) = \sum_{i}w_{i}x_{i} + b \\
\end{equation}

\paragraph{Wagi neuronu} \mbox{}\\
Połączenia w sieci realizowane są między wyjściem poprzedniego neuronu \textit{i}
oraz wejściem następnego neuronu \textit{j}. Każde takie połączenie ma przypisaną
wartość wagi \textit{w\textsubscript{ij}}. Podczas procesu uczenia wagi zmieniają
się, dostosowując sieć neuronową do otrzymywanych danych, co skutkuje
zmniejszeniem wartości funkcji błędu.

\paragraph{Bias} \mbox{}\\
Bias to dodatkowa waga wejściowa do neuronu umożliwiająca jego lepsze dopasowanie
do danych treningowych \cite{needForBias}. W sytuacji, kiedy wszystkie wagi neuronu mają zerowe
wartości, unikamy problemów podczas procesu wstecznej propagacji.

\paragraph{Warstwa} \mbox{}\\
Sieć neuronowa zorganizowana jest w warstwach. Neurony w danej warstwie nie są
ze sobą w żaden sposób połączone, komunikacja odbywa się tylko między kolejnymi
warstwami \cite{CS231n, substBigConv}. Istnieje wiele rodzajów warstw, a sygnał przechodzący przez całą
sieć zaczyna się w tzw. warstwie wejściowej oraz kończy w tzw. warstwie wyjściowej.
Istnieją sieci neuronowe (rekurencyjne sieci neuronowe, \textit{, ang. RNN - Recurrent Neural Network})
w ktorych sygnał może przechodzić przez warstwy kilkukrotnie w trakcie jednej epoki.

\paragraph{Funkcja błędu} \mbox{}\\
Funkcja błędu lub funkcja kosztu jest niezbędna do prawidłowego przeprowadzenia
procesu uczenia. Dostarcza informacje o różnicy między obecnym stanem sieci, a
optymalnym rozwiązaniem. Algorytm uczenia analizuje wartość funkcji kosztu
w kolejnych krokach w celu jej zminimalizowania.

\paragraph{Współczynnik uczenia} \mbox{}\\
Współczynnik lub wskaźnik uczenia się \textit{(ang. lerning rate)} wykorzystywany jest
w optymalizatorach do ustalenia, jak szybko algorytm uczenia powinien dostosowywać wartości
wag do danego przypadku. Często jest uzależniony od wyboru rodzaju optymalizatora. Zbyt mały
współczynnik wydłuży proces uczenia, natomiast za duży może spowodować problemy ze znalezieniem
odpowiedniego rozwiązania. W niektórych algorytmach współczynnik uczenia zmniejsza się w czasie,
by lepiej dostosować się do danych.

\section{Propagacja wsteczna}
Propagacja wsteczna lub wsteczna propagacja błędów \textit{(ang. Backpropagation)}
jest jednym z najskuteczniejszych algorytmów uczenia sieci neuronowych \cite{CS231n_backprop, backprop}. Polega
na minimalizacji funkcji kosztu, korzystając z metody najszybszego spadku lub
innych, bardziej zoptymalizowanych sposobów. Błędy w sieci propagowane są od warstwy wyjściowej
do wejściowe, czemu algorytm zawdzięcza swoją nazwę.

\section{Konwolucyjna sieć neuronowa}
\paragraph{Konwolucja} \mbox{}\\
Konwolucja, inaczej splot polega na złożeniu dwóch funkcji. W przypadku obrazów
jedna z tych funkcji to obraz, który ma rozmiary większe niż druga funkcja określana
mianem filtra konwolucyjnego. Zastosowanie splotu, w zależności od przypadku,
pozwala na rozmycie, wyostrzanie lub wydobycie głębi z danego obrazu. \textit{Szczegółowe wyjaśnienie} \cite{konwolucja}
\begin{equation}
h[m, n] = (f * g)[m, n] = \sum_j^m \sum_k^n f[j, k] * g[m-j, n-k] \\
\end{equation}

\paragraph{Konwolucyjna sieć neuronowa} \mbox{}\\
Konwolucyjna lub splotowa sieć neuronowa \textit{(ang. CNN - Convolutuinal Neural Network)}
to typ sieci odnoszący największe osiągnięcia w dziedzinie rozpoznawania obrazów,
w wielu przypadkach dorównując lub nawet pokonując ludzkie wyniki. Zawdzięczają to
swojej budowie, która różni się od zwykłych sieci wykorzystaniem warstw konwolucyjnych
i poolingowych, poprzedzających warstwy w pełni połączone. Sieć taka analizuje obraz
przy użyciu filtrów konwolucyjnych, dzięki którym jest w stanie rozpoznawać cechy
obrazów, co znacząco poprawia ich klasyfikacje.

\section{Warstwy sieci neuronowej}
\paragraph{Wejściowa} \mbox{}\\
W pracy, gdzie zbiorami danych są zbiory obrazów, każdy pojedynczy piksel obrazu
odpowiada jednej wartości liczbowej. W związku z tym rozmiar pierwszej warstwy
wejściowej jest identyczny z wymiarami obrazu. Warstwa wejściowa charakteryzuje się
brakiem wejść oraz biasu.

\paragraph{Wyjściowa} \mbox{}\\
Rozmiar warstwy wyjściowej odpowiada ilości klas, do jakiej wejściowe dane miały
zostać sklasyfikowne. Oczekiwanym wyjściem sieci w pracy była liczba oczek możliwych
do wyrzucenia na kostce, co odpowiada 6 klasom, po jednej na każdą wartość na boku
kostki. Wyjściem wszystkkich przedstawianych w tej pracy sieci był wektor o wymiarach
6x1.

\paragraph{Konwolucyjna} \mbox{}\\
Warstwa konwolucyjna służy do przetworzenia danych z poprzedniej warstwy przy użyciu
filtrów konwolucyjnych \cite{CS231n}. Filtry mają określone wymiary i służą do znajdowania cech
na obrazach lub ich fragmentach. Najczęściej spotykanymi przykładami filtrów są
kwadraty o wymiarach 3x3 piksele, które przetwarzają informacje zawarte w 9 pikselach
na jeden piksel wyjściowy.
Zastosowanie wielu warstw konwolucyjnych umożliwia filtrom analizowanie bardziej złożonych
zależności na obrazach i jest określane jako głęboka sieć. Szeroka sieć posiada większą
liczbę neuronów w każdej z warstw, co umożliwia precyzyjniejszą obserwację danych.
Ograniczeniem w przypadku sieci głębokiej i szerokiej jest ilość i czas obliczeń, co
wymusza wybranie kompromisu między ilością warstw i neuronów dla danego problemu.

\paragraph{Aktywacyjna} \mbox{}\\
Jest to wydzielenie funkcji aktywacji do osobnej warstwy, które jest realizowane
w niektórych bibliotekach. Celem takiego zabiegu jest umożliwienie podglądu danych
na wyjściu neuronu, tuż przed zaaplikowaniem samej funkcji aktywacji.

\paragraph{W pełni połączona} \mbox{}\\
Sieć neuronowa składa się z w pełni połączonych warstw \textit{(ang. Fully Connected, Dense)}.
W konwolucyjnych sieciach neuronowych warstwy te występują po warstwach konwolucyjnych
i służą do powiązania nieliniowych kombinacji, które zostały wygenerowane przez
warstwy konwolucyjne oraz ich sklasyfikowania. Dodatkowo nie wymagają dużych nakładów
obliczeniowych i są stosunkowo proste do zaaplikowania. Swoją nazwę biorą od sposobu, w jaki
realizowane sa połączenia między warstwami. Każdy neuron łączy się ze wszystkimi neuronami
następnej warstwy.

\paragraph{Flatten} \mbox{}\\
Warstwa spłaszczająca \textit{(ang. Flatten)} stosowana jest w celu połączenia warstw
konwolucyjnych lub aktywacji wraz z warstwami w pełni połączonymi. Realizowane jest
to poprzez przekształcenie warstwy wejściowej do jednowymiarowego wektora, który następnie
służy za wejście do kolejnych warstw.

\paragraph{Odrzucająca} \mbox{}\\
Warstwa odrzucająca \textit{(ang. Dropout)} zapobiega przetrenowaniu \textit{(ang. Overfitting)}
sieci \textit{Zapobieganie przetrenowaniu} \cite{DropoutPreventOverfit}.
Proces ten polega na nie wykorzystywaniu wyjść pewnych neuronów, zarówno
w przypadku przechodzenia w przód oraz w tył \textit{DropConnect} \cite{DropConnect}. Stosuje się ją po warstwach w pełni
połączonych, w celu zapobiegania rozległym zależnościom między neuronami. W warstwie
tej określone jest prawdopodobieństwo \textit{p} z jakim neuron zostanie zachowany
w warstwie oraz \textit{p - 1} z jakim zostanie odrzucony. Najczęstsza wartość jest
z zakresu 0,5-0,8.

\paragraph{Pooling} \mbox{}\\
Warstwa tzw. poolingu wykorzystywana jest do zmniejszenia rozmiaru pamięci oraz ilości obliczeń
wymaganych przez sieć neuronową, jak również może zapobiegać przetrenowaniu \cite{CS231n}. Operacja zmniejszenia
polega na wybraniu jednego piksela z danego obszaru i przekazaniu go dalej. Najczęściej wykorzystywaną
warstwą poolingową jest MaxPooling, wybierający piksel o największej wartości. Obszar, z jakiego
wybieramy dany piksel, zależy od ustawień, najczęściej jest to kwadrat o wymiarach 2x2. Pooling
jest krytykowany, ponieważ nie zachowuje informacji o położeniu piksela przekazanego na wyjście
warstwy, co może objawiać się błędnymi interpretacjami podczas testowania sieci.

\section{Funkcje aktywacji}
\paragraph{Funkcja aktywacji} \mbox{}\\
Przy pomocy funkcji aktywacji obliczana jest wartość wyjściowa neuronów w sieci
neuronowej. Argumentem dostarczanym do funkcji aktywacji jest suma wejść neuronu
pomnożonych przez przypisane im wartości wag. Zależnie od konkretnego rodzaju funkcji
aktywacji, neuron po przekroczeniu danego progu wysyła sygnał wyjściowy, odbierany
przez neurony znajdujące się w następnej warstwie. Jeśli próg nie zostanie przekroczony,
neuro nie wyśle żadnego sygnału.
\begin{equation}
f\Big(\sum_{i}w_{i}x_{i} + b\Big) \\
\end{equation}

\paragraph{Liniowa} \mbox{}\\
Funkcja ta jest praktycznie niewykorzystywana w sieciach neuronowych \cite{activationFunctions}.
Połączenie wielu warstw, których neurony mają liniową funkcję aktywacji można przedstawić
za pomocą jednej warstwy, ponieważ złożenie wielu funkcji liniowych również będzie
funkcją liniową. Nieliniowość funkcji pozwala na klasyfikację danych przechodzących
przez sieć.
\begin{equation}
f(x) = x
\end{equation}

\paragraph{Sigmoid} \mbox{}\\
Największym problemem funkcji sigmoidalnej jest duże ryzyko zaniknięcia gradientu,
co może prowadzić do problemu tzw. umierającego neuronu \textit{Opis Sigmoid} \cite{activationFunctions}.
Zjawisko to ma miejsce, gdy dla danej funkcji aktywacji, gradient staję się bardzo mały, co jest równoznaczne
z zaprzestaniem procesu uczenia. W przypadku tej funkcji gradient może zanikać obustronnie.
\begin{equation}
f(x) = \sigma(x) = \frac{1}{1 + e^{-x}} \\
\end{equation}

\paragraph{Tangens hiperboliczny} \mbox{}\\
Tangens hiperboliczny lub tanh jest w istocie przekształconą funkcją simgodalną
\textit{Opis tanh} \cite{activationFunctions, activationFunctionsV2}. Wykorzystanie
jej powoduje większe wahania gradientu.
\begin{equation}
f(x) = tanh(x) = \frac{(e^x - e^{-x})}{(e^x + e^{-x})} = \frac{2}{1+e^{-2x}} - 1 = 2 sigmoid(2x) - 1\\
\end{equation}

\paragraph{ReLU} \mbox{}\\
ReLU \textit{(and. Rectified linear unit)} jest najpopularniejszą funkcją aktywacji
wykorzystywaną w sieciach neuronowych \textit{Wady i zalety ReLU} \cite{CS231n_activ, WIKIrectifier}.
Zasługą tego jest szybki czas uczenia sieci
bez znaczącego kosztu w postaci generalizacji dokładności. Problem z zanikającym
gradientem jest mniejszy niż w przypadku funkcji sigmoidalnej, ponieważ występuje
on tylko z jednej strony.
\begin{equation}
f(x) = max(0, x)
\end{equation}

\paragraph{LeakyReLU} \mbox{}\\
LeakyReLU jest ulepszeniem ReLU \textit{Wady i zalety LeakyReLU} \cite{CS231n_activ}
dzięki zastosowaniu niewielkiego gradientu w sytuacji, dla której ReLU jest nieaktywne. Zmiana ta pozwala na uniknięcie problemu tzw
umierającego neuronu.
\begin{equation}
f(x) =
\begin{cases}
x & \text{if } x \geqslant 0 \\
0.01x & \text{if } x < 0 \\
\end{cases}
\end{equation}

\section{Optymalizatory}
\paragraph{Optymalizator} \mbox{}\\
Inne określenie algorytmu optymalizacyjnego wykorzystywanego do obliczania wag neuronów
i biasu sieci neuronowej. Posiada kluczowe znaczenie podczas procesu uczenia sieci,
zarówno w kwestii czasu oraz skuteczności. Z tego powodu jest to jeden z kluczowych
obszarów obecnych badań i rozwoju sieci neuronowych \textit{Artykuł na temat optymalizatorów} \cite{typesOfOptimizationAlgorithms}.

\paragraph{Metoda gradientu prostego} \mbox{}\\
Metoda gradientu prostego \textit{(ang. Gradient Descent)} jest podstawowym algorytmem
służacym do uaktualniania wartości wag oraz biasu podczas procesu uczenia sieci. Wadą tej
metody jest przeprowadzanie jednorazowej aktualizacji po wyliczeniu gradientu dla
całego zestawu danych. Jest to bardzo powolne, w niektórych przypadkach może powodować
problem z ilością zajmowanego miejsca w pamięci. Największą wadą jest możliwość doprowadzenia
do stagnacji w jednym z lokalnych minimów funkcji.
\begin{equation}
\theta = \theta - \eta * \nabla J(\theta)
\end{equation}

\paragraph{Stochastic Gradient Descent} \mbox{}\\
\textit{(nazwa w języku angielskim z powodu braku znalezienia polskiego odpowiednika)}
Stochastic Gradient Descent \textit{(skrót. GDA)} jest rozwinięciem metody gradientu
prostego, bardzo często wykorzystywana w praktyce \textit{Opis SGD} \cite{OptimizersOverview}.
Ulepszenie polega na obliczaniu gradientu dla jednego lub niewielkiej ilości przykładów
treningowych. Najczęściej korzysta się z więcej niż jednego przykładu, co zapewnia
lepszą stabilność oraz wykorzystuje zrównoleglenie obliczeń. SGD zapewnia większą rozbieżność
niż metoda gradientu prostego, co umożliwia znajdowanie nowych lokalnych minimów, ale
wiąże się z koniecznością zastosowania mniejszego współczynnika uczenia.
\begin{equation}
\theta = \theta - \eta * \nabla J(\theta; x_i; y_i)
\end{equation}

\paragraph{RMSprop} \mbox{}\\
RMSprop umożliwia obliczanie gradientu dla każdego parametru z osobna i zapobiega
zmniejszaniu się współczynnika uczenia \textit{Opis RMSprop} \cite{RMSpropOptimizer, OptimizersOverview}.
Algorytm dostosowuje współczynnik uczenia dla każdej wagi,
bazując na wielkości jej gradientu.

\paragraph{Adam} \mbox{}\\
Adam to skrót od angielskiej nazwy \textit{Adaptive Moment Estimation} i jest rozwinięciem
metody Stochastic Gradient Descent \textit{Opis metody Adam} \cite{AdamOptimizer, OptimizersOverview}.
Metoda ta pozwala na obliczanie z osobna gradientu dla
każdego parametru oraz każdej zmiany momentum. Zapobiega dodatkowo zmniejszającemu się
wskaźnikowi uczenia, a co najważniejsze jest bardzo szybka i pozwala na sprawne uczenie
się sieci. W tej metodzie oblicza się dwa momenty \textit{m} oraz \textit{v}.
\begin{equation}
\begin{align*}
\hat{m_t} = \frac{m_t} {1 - \beta^t_1}, \\
\hat{v_t} = \frac{v_t} {1 - \beta^t_2}, \\
\end{align*}
\end{equation}
Obliczone momenty podstawiane są do wzoru
\begin{equation}
\theta_{t+1} = \theta_t - \frac {\eta} {\sqrt{\hat{v_t}} + \epsilon} \hat{m_t}
\end{equation}
gdzie najczęściej \textbeta \textsuperscript{t}\textsubscript{1} = 0.9 oraz
\textbeta \textsuperscript{t}\textsubscript{2} = 0.99 a \straightepsilon = 10^{-8}\\

\section{Procesy}

\paragraph{Uczenie} \mbox{}\\
Proces uczenia bądź treningu sieci służy zmianie wartości wag, najczęściej zainicjowanych
pseudolosowymi wartościami oraz biasu. Uczenie sieci neuronowej jest bardzo
kosztowne obliczeniowo, co wręcz uniemożliwiało trenowanie modeli w przeszłości,
a obecnie jest jednym z powodów dużego zainteresowania rozwojem technologicznym kart
graficznych. Operacje dodawania oraz mnożenia wektorów i macierzy wykonywane są miliony razy,
mogą być przyśpieszone dzięki możliwościom zrównoleglenia obliczeń.\\

\paragraph{Epoka} \mbox{}\\
Proces uczenia sieci podzielony jest na epoki. Każda epoka odpowiada przejściu
wszystkich elementów z treningowego zbioru danych przez sieć. Ilość epok, podczas których sieć będzie się uczyć ustala się na co najmniej kilkanaście. W przypadku
większych zbiorów danych lub większych modeli ilość epok jest zwiększana.\\
Często spotykaną praktyką w wielu pracach naukowych jest przedstawianie wyników
dla sieci po 100 epokach treningu.

\paragraph{Testowanie} \mbox{}\\
Model sieci neuronowej poddawany ocenie, dzięki której można określić, w jakim stopniu
prawidłowo rozpoznaje obrazy. W przypadku wytrenowanych modeli istotne jest, aby zbiór służący do
testowania nie był wcześniej użyty do treningu sieci. Nauczony model powinien być w stanie
rozpoznawać nowe, nieużyte podczas procesu uczenia dane i poprawnie je klasyfikować.

\paragraph{Predykcja} \mbox{}\\
Wartości zwrócone przez sieć po umieszczeniu w niej określonych danych są określane
mianem predykcji. Pozwala to na wykorzystanie nauczonego modelu w praktycznym
zastosowaniu.

\section{Inne}

\paragraph{ILSVRC} \mbox{}\\
ImageNet Large Scale Visual Recognition Competition \cite{ILSVRC} to coroczny konkurs organizowany
od 2010 roku, w którym naukowcy walczą o najlepszy wynik w dziedzinie rozpoznawania
obrazów przez skonstruowane przez siebie algorytmy. Sieć AlexNet, na której bazowało
kilka stworzonych sieci neuronowych, została stworzona na potrzeby tego konkursu
w 2012 roku.

\paragraph{MNIST} \mbox{}\\
Baza danych MNIST \cite{MNIST} to zbiór 60000 treningowych i 10000 testowych czarno-białych obrazów
w rozmiarze 28x28x1, zawierających ręcznie napisane cyfry 0-9. Jest jednym z
najpopularniejszych zbiorów służących do rozpoczęcia nauki sztucznych sieci neuronowych
i uczenia maszynowego.

\paragraph{CIFAR-10} \mbox{}\\
Zbiór CIFAR-10 \cite{CIFAR-10} składa się z 50000 treningowych i 10000 testowych kolorowych obrazów w rozmiarze
32x32x2. Jest podzielony na 10 klas: samolot, samochód, ptak, kot, jeleń, pies, żaba,
koń, statek, ciężarówka; gdzie każdej z nich przypada po 6000 obrazów. Jest to podobnie
jak MNIST jeden z najbardziej popularnych zbiorów do uczenia maszynowego i sieci neuronowych.

\\\\\\
Tematy do uwzględnienia:
opis RMSprop,
minibatch,
nesterov,
momentum,
softmax,
categorical crossentropy,
one-hot encoding,


% !TEX encoding = UTF-8 Unicode

\chapter{Tworzenie zbiorów danych}
Priorytetem było stworzenie zestawu danych
umożliwiającego rozwój różnych modeli sieci neuronowych bez konieczności każdorazowego
ingerowania w zbiór lub dopasowywania go specjalnie do konkretnej sieci.\\
\textbf{Wszystkie zbiory dostępne są pod linkami w części Dodatek A [\ref{dodatekA}] }

\section{Zbiór obrazów kwadratowych}
Pierwszym założeniem było zrobienie określonej ilości zdjęć kości położonej na środku
obrazu otrzymywanego z kamery, na obszarze kwadratu o boku około 3 krotnie większym
niż długość ściany samej kości. Miało to umożliwić kadrowanie obrazów
do niewielkich rozmiarów z kośćmi położonymi w różnych jego miejscach.\\
Kolejną kwestią była konieczność przystosowania sieci do rozpoznawania kości o
różnych kolorach ścian, oczek i samego tła. Ponieważ docelowo sieć
miała operować na rzeczywistym obrazie z kamery, ważne było także zapewnienie poprawnego
działania w przypadku niewielkich zniekształceń obrazu bądź szumów.\\
Powyższe wymagania narzuciły konkretną formę zestawów danych.
Każdy zestaw ma ustaloną barwę tła, kości oraz oczek. Każda ścianka sfotografowana jest
od 3 do 30 razy, określając liczebność każdego ze zbiorów miedzy 18 a 180 zdjęć.
Zdjęcia w niektórych zestawach poddawane były działaniu lub ostrego punktowego światła
w celu wygenerowania trudniejszych do rozpoznania obrazów. Poszczególne zestawienia
kolorystyczne w tych zbiorach wypisane są w tabeli poniżej. \newpage

\begin{table}[ht]
\centering
\begin{tabular}{rcl|cc}
 \multicolumn{3}{c}{Kolor} & \multicolumn{2}{c}{Ilość zdjęć} \\ \hline
kości & oczek & tła & ściany & kostki \\ \hline
biały & czarny & czerwony & 30 & 180 \\
biały & czarny & granatowy & 3 & 18 \\
biały & czarny & czarny & 8 & 48 \\
beżowy & czarny & czerwony & 3 & 18 \\
czarny & biały & czarny & 10 & 60 \\
czarny & biały & czerwony & 10 & 60 \\
czerwony & biały & czerwony & 5 & 30 \\
czerwony & czarny & czerwony & 3 & 18 \\
granatowy & złoty & biały & 4 & 24 \\
granatowy & złoty & niebieski & 6 & 36 \\
zielony & biały & zielony & 8 & 48 \\
zielony & biały & biały & 5 & 30 \\
różowoczerwony & czarny & biały & 5 & 30 \\ \hline
\multicolumn{3}{c}{\textit{Łączna ilość zdjęć:}} & \textit{100} & \textit{600}
\end{tabular}
\vspace{0.2cm}
\caption{Zestawienie kolorystyczne obrazów 1}
\label{tab:zestawienie1}
\end{table}

Każdy z obrazów został następnie poddany przekształceniom by zwiększyć ich ilość, konieczną
do prawidłowego uczenia się sieci. Miało to także zapewnić zniekształcone w niewielkim stopniu
zdjęcia, które również powinny być rozpoznawane przez sieć. W tym przypadku zastosowano sześć
takich przekształceń. \\
Następnym krokiem było zastosowanie obrotów o kąty 5\textsuperscript{o},
15\textsuperscript{o}, 30\textsuperscript{o} lub 45\textsuperscript{o}, w celu dobrania
najbardziej adekwatnej ilości obrazów do rozmiaru sieci oraz ilości parametrów
do nauczenia się. Po wielu próbach ostatecznie wybrano kąt 15\textsuperscript{o}, zapewniający
wystarczająco liczny zbiór treningowy do możliwości uczenia sieci oraz nie
wydlużało znacząco czasu potrzebnego do przetworzenia wszystkich obrazów.\\
Powyższe procesy spowodowały uzyskanie 168 zdjęć o rozmiarze 64x64 piksele z każdego
z początkowych obrazów o rozmiarze 1600x1200 pikseli. Cały zbiór danych liczył 100800
obrazów, zarówno w wersji kolorowej RGB oraz w odcieniach skali szarości. Części
treningowa i testowa zbioru danych zostały rozdzielone w stosunku 4:1, dając odpowiednio
80640 oraz 20160 zdjęć w każdym z nich. \\\\

\begin{figure}[h]
\centering
\includegraphics[scale=0.35]{images/kolaz}
\includegraphics[scale=0.35]{images/kolaz_grayscale}
\caption{Obrazy o rozdzielczościach 64x64}
\end{figure}

Nieocenioną pomocą w początkowej fazie prac nad tworzeniem i uczeniem sieci neuronowych
z obrazami w kształcie kwadratów był fakt, że praktycznie wszystkie przykłady dostępne
w opracowaniach naukowych korzystały ze zdjęć w tym kształcie. Również wszystkie
przykłady opisane w różnych poradnikach czy najpopularniejsze zbiory jak MNIST oraz
CIFAR10 zawierają kwadratowe obrazy. Przyjęcie takiego kształtu ułatwiło dobór parametrów
sieci, nie narzucając osobnych wartości w poziomie i pionie.

 \section{Zbiór obrazów prostokątnych}

Po nauczeniu i weryfikacji kilkunastu różnych modeli sieci, zdecydowano się podjąć
próbę z wykorzystaniem większych obrazów oraz kości rozmieszczonych w miejscach
bardziej zróżnicowanych niż jedynie pewien, niewielki obszar w centrum obrazu.
Kolejne zdjęcia miały również podnieść trudność uczenia poprzez zmniejszenie
rozmiaru kości w stosunku do rozmiaru całego obrazu. Ostatnim czynnikiem decydującym o
rozwinięciu zbiorów danych była trudność w rozpoznawaniu kości w czasie rzeczywistym
w sytuacji kiedy rozmiar obrazu lub jego wycinka to jedynie 64x64 piksele. Zwiększenie
rozmiarów ułatwiłoby identyfikacje kości oraz umożliwiło by detekcję ilości oczek
wyrzuconych na kostce o ile tylko kość znalazła by się w obszarze odejmowanym przez kamerę. \\
Plan zakładał wykorzystanie poprzednio wykonanych zdjęć oraz dodanie nowych
z kośćmi rozmieszczonymi poza obszarem w centrum obrazu w celu rozwinięcia możliwości sieci.
Jednocześnie zrodził się pomysł wykorzystania prostokątnych obrazów, które lepiej oddawałyby
rzeczywistość, gdzie prawie wszystkie kamery, niezależnie od zastosowań, dostarczają
prostokątny obraz. W tym celu, analogicznie jak w przypadku kwadratowych obrazów,
zostały wykonane zdjęcia o określonych zestawieniach kolorystycznych. W tabeli poniżej
wypisane jest ich zestawienie:

\begin{table}[h!]
\centering
\begin{tabular}{rcl|cc}
 \multicolumn{3}{c}{Kolor} & \multicolumn{2}{c}{Ilość zdjęć} \\ \hline
kości & oczek & tła & ściany & kostki \\ \hline
biały & czarny & zielony & 6 & 36 \\
czerwony & biały & zielony & 6 & 36 \\
czerwony & czarny & różowy & 7 & 42 \\
czarny & biały & szary & 6 & 36 \\
czarny & biały & niebieski & 6 & 36 \\
beżowy & czarny & szary & 6 & 36 \\
beżowy & czarny & niebieski & 6 & 36 \\
granatowy & złoty & biały & 8 & 48 \\
zielony & biały & żółty & 7 & 42 \\
różowoczerwony & czarny & pomarańczowy & 6 & 36 \\ \hline
\multicolumn{3}{c}{\textit{Łączna ilość zdjęć:}} & \textit{64} & \textit{384}
\end{tabular}
\vspace{0.2cm}
\caption{Zestawienie kolorystyczne obrazów 2}
\label{tab:zestawienie2}
\end{table}

\begin{figure}[h]
\centering
\includegraphics[scale=0.5]{color_dices}
\caption{Obrazy o rozdzielczościach 106x79}
\end{figure}

Ilość zdjęć prostokątnych jest mniejsza niż kwadratowych z powodu czasu jaki zajmuje
wykonanie takiej ilości zdjęć oraz chęć wykorzystania obu rodzajów zbiorów do uczenia sieci.
Powstała w ten sposób liczba 984 unikalnych zdjęć jest wystarczająca do realizacji
zamierzonego zadania i pozwala na odpowiednie uczenie sieci. \\
Poprzednio wykorzystywane obrazy, miały wymiary 64x64 piksele co łącznie odpowiadało
4096 lub 12288 wartościom pikseli odpowiednio dla obrazów w skali szarości oraz
kolorowych RGB. Nowo utworzony zbiór obrazów prostokątnych w pierwszym założeniu miał
składać się z obrazów o rozmiarach 320x240 pikseli w skali szarości, co oznaczało
76800 wartości na jedno zdjęcie. W wyniku niepowodzenia procesu uczenia po kilku godzinach,
podjęto decyzję o dwukrotnym zmniejszeniu obrazów do 160x120 pikseli. Wartość ta
została wybrana ponieważ ilość parametrów obrazu, wynosząca 19200, była jedynie o 56\%
większa od ich liczby dla kolorwych obrazów 64x64. Próby uczenia się sieci pokazały jednak,
że lepszym rozwiązaniem będzie zastosowanie mniejszych o 50\% obrazów o wymiarach
106x79 pikseli co skutkuje liczbą 8374 wartości na jeden obraz.\\
Wraz z problemami związanymi z rozmiarami obrazów, zdecydowano się na zmniejszenie ilości
przekształceń z sześciu do wybranych czterech, najbardziej efektowych. Inne przedstawiały
zmieniony w niewielkim stopniu obraz, niepotrzebnie wydłużając proces
uczenia. Również kąt obrotu zdjęć został zwiększony z 15\textsuperscript{o}
do 30\textsuperscript{o}, co w założeniu nie powinno powodować problemów z rozpoznawaniem
kości przy różnych ustawieniach. \\
Wszystkie operacje umożliwiły osiągnięcie 60 obrazów z każdego początkowego zdjęcia, w rozmiarze
106x79 pikseli w odcieniach skali szarości. Całkowita ilość obrazów w pełnym zbiorze
danych wynosiła 59040, co przełożyło się na 47232 obrazów treningowych i 11808 testowych,
korzystając z identycznego jak wcześniej stosunku 4:1.



\chapter {Sieć neuronowa}

\section {Czym jest sieć neuronowa}

\section {Konwolucyjna sieć neuronowa}


\end{document}
