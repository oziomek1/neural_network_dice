% !TEX encoding = UTF-8 Unicode

\documentclass[a4paper,oneside,12pt]{report}

\usepackage{titlesec}
\usepackage[top=0cm, bottom=0cm, left=2cm, right=2cm]{geometry}
\usepackage[T1]{fontenc}
\usepackage[polish]{babel}
\usepackage[utf8]{inputenc}
\usepackage{lmodern}
\selectlanguage{polish}
\makeatletter

\renewcommand{\maketitle}{\begin{titlepage}
	
	\vspace*{ \stretch{1} }
	\begin{center} \LARGE
	Uniwersytet Jagielloński \\
  	Wydział Fizyki, Astronomii i Informatyki Stosowanej
	\end{center}
	
	\vspace{ \stretch{1} }
	\begin{center} \Huge 
		\textsc {\@title}
	\end{center}
			
	\vspace{ \stretch{1} }
	\begin{center}
		\large \@author \\
		\vspace{3mm}
		\normalsize Nr albumu: 1124802
	\end{center}
	
	\vspace{ \stretch{1.5} }
	\begin{flushright}
	\begin{minipage}{8cm}
	\begin{center} \large
	Praca wykonana pod kierunkiem \\
	prof. dr hab. Piotra Białasa \\
	kierownika Zakładu Technologii Gier \\
	FAIS UJ
	\end{center}
	\end{minipage}
	\end{flushright}
	\vspace{ \stretch{1} }
	\begin{center}
	\@date
	\end{center}
\end{titlepage}
}
\makeatother

\titleformat{\chapter}[display]
  {\normalfont\bfseries}{}{0pt}{\Huge}

\renewcommand*\contentsname{Spis Treści}

\author{Wojciech Ozimek}
\title{Wykorzystanie sztucznych sieci neuronowych do analizy obrazów na przykładzie kostki do gry}
\date{kwiecień 2018}

\begin{document}
\maketitle{}

\tableofcontents
\newpage


% !TEX encoding = UTF-8 Unicode

\chapter{Wstęp}

Sieci neuronowe składają się ze sztucznych neuronów, które są uproszczonymi modelami ich
biologicznych odpowiedników. Pierwsze prace nad tym zagadnieniem zaczęły sie już w latach 50.
XX wieku.

\section{Sztuczny neuron}

Najprostszym modelem sztucznego neuronu jest tzw. neuron McCullocha-Pittsa, nazywany
również neuronem binarnym. Taki neuron posiada wiele wejść, z których każde ma przypisaną
wagę w postaci liczby rzeczywistej. Suma wartości wejściowych mnożona jest przez odpowiadające
im wagi i podawana jako argument do funkcji aktywacji, wyjaśnionej w późniejszych rozdziałach.
Wartość tej funkcji jest wyjściem neuronu i staje się wejściem innych neuronów,
tak jak ma to miejsce w przypadku biologicznego neuronu \cite{CS231n}.\\
Neurony najczęściej łączone są w struktury nazywane sztucznymi sieciami neuronowymi, ponieważ wykorzystanie
pojedynczego neuronu nie pozwala na rozwiązywanie skomplikowanych zadań.
Przykładowo próba realizacji prostych funkcji logicznych ANR, OR, NOT i XOR okazuje
się możliwa jedynie dla trzech pierwszych podanych funkcji \cite{XORproblem}.
Problem wiąże się z brakiem możliwości uzyskania poprawnych rezultatów dla zbiorów które
nie są liniowo separowalne, czego przykładem jest właśnie funkcja XOR. Tworzenie rozbudowanych
struktur z pojedynczych neuronów znacząco zwiększa możliwości adaptacyjne
dla poszczególnych problemów, niejednokrotnie zaskakując osiąganymi wynikami.

\section{Historia}

Rozpoczęcie prac nad sztucznym neuronem można datować na rok 1943 kiedy
Warren McCulloch i Walter Pitts przedstawili jego model \cite{NNbiology}.
Pierwszymi zastosowaniami sieci neuronowych były operacji bitowe oraz przewidywania kolejnych
wystąpień bitów w ciągu. Mimo licznych prób zastosowania ich do realnych problemów
nie zyskały one popularności. W początkowym okresie rozwoju sieci neuronowych
przyjmowano także wiele błędnych założeń, które wraz z ograniczonymi możliwościami
obliczeniowymi komputerów skutecznie zniechęcały naukowców do prac nad tym zagadnieniem.\\


\paragraph {Lorem  ipsum  dolor  sit  amet,  consectetuer  adipiscing  
elit.   Etiam  lobortisfacilisis sem.  Nullam nec mi et 
neque pharetra sollicitudin.  Praesent imperdietmi nec ante. 
Donec ullamcorper, felis non sodales...}


\chapter {Technologie}

\section {Język programowania i środowisko}

\subsection {Python}

\subsection {Jupyter Notebook}

\section {Biblioteki}

\subsection {OpenCV}

\subsection {Tensorflow}

\subsection {Keras}

\subsection {Numpy}

\subsection {Matplotlib}

\subsection {LaTeX}

\section {Technologie poza programistyczne}

\subsection{Nvidia CUDA}

\subsection {Amazon AWS EC2}

\subsection {Google Compute Engine}


\chapter {Założenia pracy}


\chapter {Sieć neuronowa}

\section {Czym jest sieć neuronowa}

\section {Konwolucyjna sieć neuronowa}


\end{document}
