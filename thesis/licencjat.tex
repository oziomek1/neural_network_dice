% !TEX encoding = UTF-8 Unicode

\documentclass[a4paper,oneside,12pt]{report}

\usepackage{titlesec}
\usepackage[top=1cm, bottom=2cm, left=2cm, right=2cm]{geometry}
\usepackage[T1]{fontenc}
\usepackage[polish]{babel}
\usepackage[utf8]{inputenc}
\usepackage{lmodern}
\usepackage{amssymb}
\usepackage{amsmath}

\usepackage{setspace}
\selectlanguage{polish}
\makeatletter

\renewcommand{\maketitle}{\begin{titlepage}

	\vspace*{ \stretch{1} }
	\begin{center} \LARGE
	Uniwersytet Jagielloński \\
  	Wydział Fizyki, Astronomii i Informatyki Stosowanej
	\end{center}

	\vspace{ \stretch{1} }
	\begin{center} \Huge
		\textsc {\@title}
	\end{center}

	\vspace{ \stretch{1} }
	\begin{center}
		\large \@author \\
		\vspace{3mm}
		\normalsize Nr albumu: 1124802
	\end{center}

	\vspace{ \stretch{1.5} }
	\begin{flushright}
	\begin{minipage}{8cm}
	\begin{center} \large
	Praca wykonana pod kierunkiem \\
	prof. dr hab. Piotra Białasa \\
	kierownika Zakładu Technologii Gier \\
	FAIS UJ
	\end{center}
	\end{minipage}
	\end{flushright}
	\vspace{ \stretch{1} }
	\begin{center}
	\@date
	\end{center}
\end{titlepage}
}
\makeatother

\titleformat{\chapter}[display]
  {\normalfont\bfseries}{}{0pt}{\Huge}

\renewcommand*\contentsname{Spis Treści}

\author{Wojciech Ozimek}
\title{Wykorzystanie sztucznych sieci neuronowych do analizy obrazów na przykładzie kostki do gry}
\date{kwiecień 2018}

\begin{document}
\maketitle{}

\tableofcontents
\newpage


% !TEX encoding = UTF-8 Unicode

\chapter{Wstęp}

Sieci neuronowe składają się ze sztucznych neuronów, które są uproszczonymi modelami ich
biologicznych odpowiedników. Pierwsze prace nad tym zagadnieniem zaczęły sie już w latach 50.
XX wieku.

\section{Sztuczny neuron}

Najprostszym modelem sztucznego neuronu jest tzw. neuron McCullocha-Pittsa, nazywany
również neuronem binarnym. Taki neuron posiada wiele wejść, z których każde ma przypisaną
wagę w postaci liczby rzeczywistej. Suma wartości wejściowych mnożona jest przez odpowiadające
im wagi i podawana jako argument do funkcji aktywacji, wyjaśnionej w późniejszych rozdziałach.
Wartość tej funkcji jest wyjściem neuronu i staje się wejściem innych neuronów,
tak jak ma to miejsce w przypadku biologicznego neuronu \cite{CS231n}.\\
Neurony najczęściej łączone są w struktury nazywane sztucznymi sieciami neuronowymi, ponieważ wykorzystanie
pojedynczego neuronu nie pozwala na rozwiązywanie skomplikowanych zadań.
Przykładowo próba realizacji prostych funkcji logicznych ANR, OR, NOT i XOR okazuje
się możliwa jedynie dla trzech pierwszych podanych funkcji \cite{XORproblem}.
Problem wiąże się z brakiem możliwości uzyskania poprawnych rezultatów dla zbiorów które
nie są liniowo separowalne, czego przykładem jest właśnie funkcja XOR. Tworzenie rozbudowanych
struktur z pojedynczych neuronów znacząco zwiększa możliwości adaptacyjne
dla poszczególnych problemów, niejednokrotnie zaskakując osiąganymi wynikami.

\section{Historia}

Rozpoczęcie prac nad sztucznym neuronem można datować na rok 1943 kiedy
Warren McCulloch i Walter Pitts przedstawili jego model \cite{NNbiology}.
Pierwszymi zastosowaniami sieci neuronowych były operacji bitowe oraz przewidywania kolejnych
wystąpień bitów w ciągu. Mimo licznych prób zastosowania ich do realnych problemów
nie zyskały one popularności. W początkowym okresie rozwoju sieci neuronowych
przyjmowano także wiele błędnych założeń, które wraz z ograniczonymi możliwościami
obliczeniowymi komputerów skutecznie zniechęcały naukowców do prac nad tym zagadnieniem.\\


% !TEX encoding = UTF-8 Unicode

\chapter{Technologie}
Rozwój informatyki i wzrost zainteresowania sieciami neuronowymi spowodowały pojawienie
się wielu, ułatwiających pracę, narzędzi. Wykorzystane w tej pracy technologie
pozwoliły na skupienie się wokół faktycznego problemu i pominięcie wielu technicznych
aspektów.

\section{Język programowania i środowisko}

\paragraph{Python} \mbox{}\\
Kod stworzony na potrzeby tej pracy został napisany w jezyku programowania Python, który
obecnie jest bardzo popularnym narzędziem wykorzystywanym w pracy naukowej.
Podowem jego tego wyboru jest bardzo czytelna i zwięzła składnia,
która w pełni pozwala skupić się na danym problemie. Python jest on także domyślnym
językiem używany w bibliotekach wykorzystywanych do tworzenia modeli sieci neuronowych.

\paragraph{Jupyter Notebook} \mbox{}\\
Do łatwiejszego tworzenia i analizowania modeli sieci neuronowych wykorzystano aplikację
Jupyter Notebook, pozwalającą uruchamiać w przeglądarce pliki, nazywane notebookami,
które składają się z wielu bloków. W blokach może znajdować się wykonywalny kod programu
lub jego fragment, a w pozostałych można prezentować m.in. teksty, wykresy bądź tabele,
co zdecydowanie ułatwia pracę.\\

\section{Biblioteki}

\paragraph{OpenCV} \mbox{}\\
Posiadanie mniejszych niż otrzymywanych bezpośrednio z kamery obrazów bardzo przyśpiesza
naukę sieci neuronowych. Do obróbki obrazów wykorzystano bibliotekę OpenCV.
Najczęsciej wykorzystywana jest w językach C++ oraz Python, pozwalając m.in. na
skalowanie, kadrowanie i obrazanie obrazów.

\paragraph{TensorFlow} \mbox{}\\
TensorFlow jest biblioteką ułatwiającą tworzenie sieci neuronowych. Jej ogromną zaletą
jest implementacja w języku C++ umożliwiająca wykorzystywanie procesorów oraz kart graficznych.
Z uwagi na charakter  wykonywanych obliczeń praca przy użyciu kart graficznych jest
kilkukrotnie szybsza niż na procesorze, co znacząco przyśpiesza proces uczenia.
Biblioteka najlepiej wspiera języki programowania C++ oraz Python.

\paragraph{Keras} \mbox{}\\
Wykorzystywaną w tej pracy biblioteką do tworzenia modeli sieci neuronowych jest Keras,
wykorzystujący bardziej niskopoziomowe biblioteki jak TensorFlow.
Zaletami Kerasa są zarówno przystępny interfejs pozwalający w krótkim czasie stworzyć model
sieci oraz możliwość tworzenia zaawansowanych modeli przy średnim pogorszeniu czasu
uczenia się sieci o 3-4\% w stosunku do bibliotek, na których bazuje.\\
W sytuacji kiedy interfejs oraz dokumentacja do TensorFlow mogą być dla początkującej
osoby niezrozumiałe, jest to świetna alternatywa do wdrożenia się w to zagadnienie.

\paragraph{Numpy} \mbox{}\\
Wykonywanie operacji na dużych zbiorach danych dostarczanych do sieci zostało przyśpieszone
dzięki wykorzystaniu modułu Numpy. Moduł umożliwia wykonywanie zaawansowanych operacji na macierzach
oraz wektorach, wspierający liczne funkcje matematyczne. Jest bardzo rozpowszechniony
i wykorzystywany w wielu, głównie naukowych zastosowaniach. Numpy wprowadza własne
typy danych oraz funkcje niedostępne w standardowej instalacji Pythona.

\paragraph{Matplotlib} \mbox{}\\
Wykresy potrzebne do analizy sieci neuronowych tworzone były przy użyciu narzędzia Matplotlib.
Posiada bardzo duże możliwości pozostając jednocześnie prostym w użyciu co sprawia
jest bardzo popularny. Dodatkowo zawiera moduł pyplot, który w założeniu ma maksymalnie przypominać
interfejs w programie MATLAB.

\paragraph{LaTeX} \mbox{}\\
Oprogramowanie do składu tekstu, nie będące edytorem tekstowy typu WYSIWYG.
LaTeX bazuje na TeX, który jest systemem składu drukarskiego do prezentacji w
formie graficznej. Ogromną zaletą jest możliwość tworzenia w tekście zaawansowanych
wzorów matematycznych. Tekst niniejszej pracy napisany został przy pomocy tego narzędzia.

\section{Technologie wspomagające}
Kilkukrotne przyśpieszenie tworzenia modeli do tej pracy możliwe było dzieki wykorzystaniu
możliwości oferowanych przez szybkie karty graficzne.

\paragraph{NVIDIA CUDA} \mbox{}\\
Technologia obecna w kartach firmy NVIDIA to równoległa architektura obliczeniowa,
pozwalająca na wielokrotne przyśpieszneie obliczeń podczas uczenia się sieci neuronowych.
Dzięki bibliotekom TensorFlow i Keras, które wspierają obliczenia na kartach graficznych,
czas uczenia sieci drastycznie maleje. Podczas tej pracy wykorzystana została
karta NVIDIA TESLA K80 12GB GPU dostępna na Amazon AWS oraz Google Compute Engine.

\paragraph{Amazon AWS EC2} \mbox{}\\
Szybkie karty graficzne wykorzytane w tej pracy były dostępne na plaftormie z
wirtualnymi maszynami, które można dostosować do potrzeb klienta. Wykorzystano maszyny
zoptymalizowane do obliczeń na kartach graficznych i uczenia maszynowego.
Dzięki wykorzystaniu Amazon AWS, czas uczenia sieci zmniejszył się średnio 6-10 krotnie
w stosunku do pracy na komputerze wykorzystującym procesor.



\chapter {Założenia pracy}


\chapter {Sieć neuronowa}

\section {Czym jest sieć neuronowa}

\section {Konwolucyjna sieć neuronowa}


\end{document}
