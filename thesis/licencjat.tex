% !TEX encoding = UTF-8 Unicode

\documentclass[a4paper,oneside,12pt]{report}

\usepackage{titlesec}
\usepackage[top=1cm, bottom=2cm, left=2cm, right=2cm]{geometry}
\usepackage[T1]{fontenc}
\usepackage[polish]{babel}
\usepackage[utf8]{inputenc}
\usepackage{lmodern}
\usepackage{amssymb}
\usepackage{amsmath}

\usepackage{setspace}
\selectlanguage{polish}
\makeatletter

\renewcommand{\maketitle}{\begin{titlepage}

	\vspace*{ \stretch{1} }
	\begin{center} \LARGE
	Uniwersytet Jagielloński \\
  	Wydział Fizyki, Astronomii i Informatyki Stosowanej
	\end{center}

	\vspace{ \stretch{1} }
	\begin{center} \Huge
		\textsc {\@title}
	\end{center}

	\vspace{ \stretch{1} }
	\begin{center}
		\large \@author \\
		\vspace{3mm}
		\normalsize Nr albumu: 1124802
	\end{center}

	\vspace{ \stretch{1.5} }
	\begin{flushright}
	\begin{minipage}{8cm}
	\begin{center} \large
	Praca wykonana pod kierunkiem \\
	prof. dr hab. Piotra Białasa \\
	kierownika Zakładu Technologii Gier \\
	FAIS UJ
	\end{center}
	\end{minipage}
	\end{flushright}
	\vspace{ \stretch{1} }
	\begin{center}
	\@date
	\end{center}
\end{titlepage}
}
\makeatother

\titleformat{\chapter}[display]
  {\normalfont\bfseries}{}{0pt}{\Huge}

\renewcommand*\contentsname{Spis Treści}

\author{Wojciech Ozimek}
\title{Wykorzystanie sztucznych sieci neuronowych do analizy obrazów na przykładzie kostki do gry}
\date{kwiecień 2018}

\begin{document}
\maketitle{}

\tableofcontents
\newpage


% !TEX encoding = UTF-8 Unicode

\chapter{Wstęp}

\section{Biologiczny neuron}

Neuron to komórka nerwowa zdolna do przewodzenia i przetwarzania sygnału
elektrycznego w którym zawarta jest informacja. Jest on podstawowym elementem
układu nerwowego wszystkich zwierząt. Każdy neuron składa się z
ciała komórki (soma, neurocyt) otaczającego jądro komórkowe, neurytu (akson)
odpowiedzialny za przekazywanie informacji z ciała komórki do kolejnych neuronów
oraz dendrytów służących do odbierania sygnałów i przesyłaniu ich do ciała
komórkowego. Impuls elektryczny z jednego neuronu do drugiego przekazywany jest
w synapsie, miejscu komunikacji danego neuronu z poprzednim. Synapsa składa się
z części presynaptycznej (aksonu) i postsynaptycznej (dendrytu). Neuron
przewodzi sygnał tylko w sytuacji kiedy suma potencjałów na wejściach od innych
neuronów na jego dendrytach przekroczy określony poziom. W przeciwnym wypadku
neuron nie przewodzi sygnału. Dodatkowo, zwiększenie potencjału na wejściach
nie powoduje wzmocnienia potencjału na wyjściu neuronu. \\
Neurony połączone i działające w ten sposób tworzą sieci neuronowe, których
dobrym przykładem może być mózg człowieka. Przeciętnie posiada on około 100
miliardów neuronów, każdy z nich połączony jest z około 10 tysiącami innych
neuronów przez połączenia synaptyczne. Liczba połączeń synaptycznych szacowana
jest na około 10\textsuperscript{15}.

\section{Sztuczny neuron}

Matematycznym modelem neuronu jest tzw. neuron McCullocha-Pittsa, nazywany
również neuronem binarnym. Jest to prosta koncepcja zakładająca, że każdy neuron
posiada wiele wejść, z których każde ma przypisaną wagę w postaci liczby
rzeczywistej oraz jedno wyjście. Wyjściem neuronu jest wartość funkcji aktywacji
dla argumentu którym jest suma wszystkich wag pomnożonych przez odpowiednie
wartości wejściowe. Wyjście danego neuronu połączone jest z wejściami innych
neuronów, tak jak ma to miejsce w przypadku biologicznego neuronu. \\
Powyższy, sztuczny model neuronu jest podstawowym budulcem sztucznych sieci
neuronowych z racji prostoty działania i łatwości implementacji.
W rzeczywistych zastosowaniach używane są pojecia takie jak wektor wejściowy
oraz wektor wag, które oznaczają odpowiednio wszystkie wartości wejścioweoraz
wszystkie wagi dla danego neuronu. \\
Pojedynczyn neuron pozwala na uzyskanie mocno ograniczonych rozwizań. Przykładowo
korzystając z prostych funkcji logicznych ANR, OR, NOT, XOR okazuje się że
pojedynczy neuron jest w stanie poprawnie rozwiązywać jedynie pierwsze trzy z
podanych funkcji. Problem wiąże się z brakiem możliwości uzyskania poprawnych
rezultatów dla zbiorów które nie są liniowo separowalne. W takich przypadkach
konieczne jest użycie większej ilości neuronów, a więc tworzenie sieci neuronowych
których możliwości adaptacyjnego do poszczególnych problemów są niejednokrotnie
bardzo zaskakujące.


\section{Historia}

Rozpoczęciem prac nad sztucznymi neuronami można datować na rok 1943 kiedy to
Warren McCulloch i Walt er Pitts przedstawili wspomniany już wcześniej model
neuronu. Pierwsze sieci neurone używane były np do operacji bitowych i
przewidywania kolejnych wystąpień bitów w ciągu. Mimo licznych prób zastosowania
ich do realnych problemów nie zyskały one popularności. Powodem tego były prace
naukowe sugerujące ograniczenia sieci takie jak brak możliwości rozszerzenia sieci
do więcej niż jednej warstwy oraz jedynie jednokierunkowe połączenie między
neuronami. W początkowym okresie rozwoju sieci neuronowych przyjmowano także wiele
błędnych założeń które wraz z ograniczonymi możliwościami obliczeniowymi komputerów
skutecznie zniechęcały naukowców do prac nad tym zagadnieniem. \\
Przełomem okazał sie rok 1982 kiedy to John Hopfield przedstawił sieć asosjacyjną
(zwaną siecią Hopfielda). Nowością było dwukierunkowe połączenie neuronów co
zapewniało możliwość uczenia się danych wzorców. Kolejnymi przełomowymi odkryciami
były zarówno wprowadzenie wielowarstwowych sieci neuronowych oraz wstecznej
propagacji. Nowe odkrycia pozwoliły na zastosowanie sieci w wielu różnych
dziedzinach, wymagając jednak dużych ilości obliczeń, obszernych zbiorów
treningowych, wielu tysięcy iteracji i długiego czasu nauki. Korzyścią jest jednak
fakt, że wytrenowany model można wykorzystać w praktycznie każdych warunkach
i natychmiastowo bez konieczności uczenia.


% !TEX encoding = UTF-8 Unicode

\chapter{Technologie}
\section{Język programowania i środowisko}

\paragraph{Python} \mbox{}\\
Język programowania Python obecnie jest bardzo popularnym narzędziem wykorzystywanym
w pracy naukowej. Jest to spowodowane bardzo czytelną i zwięzłą składnią,
która w pełni pozwala skupić się na danym problemie. Python jest on także domyślnym
językiem używany w bibliotekach wykorzystywanych do tworzenia modeli sieci neuronowych.

\paragraph{Jupyter Notebook} \mbox{}\\
Aplikacja Jupyter Notebook pozwala uruchamiać w przeglądarce pliki, nazywane notebookami,
które składają się z wielu bloków. W blokach może znajdować się wykonywalny kod programu
lub jego fragment, a w pozostałych można prezentować m.in. teksty, wykresy bądź tabele,
co zdecydowanie ułatwia pracę.\\
W tej pracy wykorzystywana jest to uruchamiania programów w języku Python.

\section{Biblioteki}

\paragraph{OpenCV} \mbox{}\\
Biblioteka funkcji do obróbki obrazów, najczęsciej wykorzystywana w językach C++
oraz Python. W projekcie została użyta do uzyskiwania zbiorów obrazów kości do gry
ze zdjęć wykonanych kamerą.

\paragraph{TensorFlow} \mbox{}\\
Biblioteka do uczenia maszynowego oraz tworzenia sieci neuronowych. Jej ogromną
zaletą jest implementacja w języku C++ umożliwiająca wykorzystywanie procesorów oraz
kart graficznych. Z uwagi na charakter wykonywanych obliczeń praca przy użyciu
kart graficznych jest kilkukrotnie szybsza niż na procesorze, co znacząco przyśpiesza
proces uczenia. Biblioteka najlepiej wspiera języki programowania do C++ oraz Python.

\paragraph{Keras} \mbox{}\\
Biblioteka do tworzenia modeli sieci neuronowych, wykorzystująca bardziej niskopoziomowe
biblioteki jak TensorFlow, Theano lub CNTK. Zaletami Kerasa są
zarówno przystępny interfejs pozwalający w krótkim czasie stworzyć model sieci
oraz możliwość tworzenia zaawansowanych modeli przy średnim pogorszeniu czasu
uczenia się sieci o 3-4\% w stosunku do bibliotek, na których bazuje.\\
W sytuacji kiedy interfejs oraz dokumentacja do TensorFlow mogą być dla początkującej
osoby niezrozumiałe, jest to świetna alternatywa do wdrożenia się w to zagadnienie.

\paragraph{Numpy} \mbox{}\\
Moduł języka Python umożliwiający wykonywanie zaawansowanych operacji na macierzach
oraz wektorach, wspierający liczne funkcje matematyczne. Jest bardzo rozpowszechniony
i wykorzystywany w wielu, głównie naukowych zastosowaniach. Numpy wprowadza własne
typy danych oraz funkcje niedostępne w standardowej instalacji Pythona.

\paragraph{Matplotlib} \mbox{}\\
Narzędzie do tworzenia wykresów dla języka Python oraz modułu Numpy. Z uwagi na
bardzo duże możliwości jest bardzo popularny, pozostając jednocześnie prostym w
użyciu. Zawiera moduł pyplot, który w założeniu ma maksymalnie przypominać interfejs
w programie MATLAB.

\paragraph{LaTeX} \mbox{}\\
Oprogramowanie do składu tekstu, nie będące edytorem tekstowy typu WYSIWYG.
LaTeX bazuje na TeX, który jest systemem składu drukarskiego do prezentacji w
formie graficznej. Ogromną zaletą jest możliwość tworzenia w tekście zaawansowanych
wzorów matematycznych. Tekst niniejszej pracy napisany został przy pomocy tego narzędzia.

\section{Technologie wspomagające}

\paragraph{NVIDIA CUDA} \mbox{}\\
Równoległa architektura obliczeniowa firmy NVIDIA pozwala na wielokrotne
przyśpieszneie obliczeń podczas uczenia się sieci neuronowych. Dzięki bibliotekom
takim jak TensorFlow lub Keras, które wspierają obliczenia na kartach graficznych
czas precesu uczenia drastycznie maleje. Podczas tej pracy wykorzystana została
karta NVIDIA TESLA K80 12GB GPU dostępna na Amazon AWS oraz Google Compute Engine.

\paragraph{Amazon AWS EC2} \mbox{}\\
Platforma z wirtualnymi maszynami zwanymi instancjami, które można dostosować
zależnie od potrzeb klienta. W tej pracy zostały wykorzystane instancje
zoptymalizowane do obliczeń na kartach graficznych i uczenia maszynowego.
Dzięki jej wykorzystaniu, czas uczenia sieci zmniejszył się średnio 6-10 krotnie
w stosunku do pracy na komputerze wykorzystującym procesor.



\chapter {Założenia pracy}


\chapter {Sieć neuronowa}

\section {Czym jest sieć neuronowa}

\section {Konwolucyjna sieć neuronowa}


\end{document}
