% !TEX encoding = UTF-8 Unicode

\chapter{Słownik pojęć}
\section{Zbiór danych}
Do realizacji pracy konieczny było stworzenie zbioru danych składającego się z obrazów
z kośćmi do gry. Zdjęcia zostały zrobione kamerą o rozdzielczości 1600x1200 pikseli,
a następnie zmniejszone w celu zaoszczędzenia pamięci i osiągnięcia żądanego rozmiaru,
który ma kluczowe znaczenie w przypadku zastosowań sieci neuronowych.
Oprócz zmniejszenia obrazy były również obracane, przekrzywiane(deformowane? wypaczane? \textit{warping}) oraz kadrowane.
Tak powstałe, liczne zbiory były wykorzystywane w trybie RGB oraz w odcieniach skali szarości,
co pozwoliło na oszczędzenie pamięci przez zmniejszenie liczby kanałów do jednego.\\
W pracy używano zbiorów z obrazami o rozmiarach 64x64 oraz 106x79 pikseli. Każde zdjęcie
w zbiorze miało przypisaną wartość liczbową informującą o faktycznej ilości oczek wyrzuconych
na przedstawionej kostce. Wartość ta zwana jest odpowiedzią i jest wykorzystywana w
procesie uczenia sieci jako docelowa informacja, którą ma zwrócić sieć po weryfikacji danego obrazu.

\paragraph{Zbiór treningowy} \mbox{}\\
Część zbioru danych, która wykorzystywana jest w procesie uczenia sieci określana jest jako zbiór
treningowy lub zbioru uczący. Jego liczebność to zazwyczaj 60-80\% całego zbioru danych.
Praktycznie zawsze dane znajdujące się w tym zbiorze, przed rozpoczęciem uczenia,
poddawane są losowej permutacji.

\paragraph{Zbiór testowy} \mbox{}\\
Do oceny zdolności sieci neuronowej do rozpoznawania danych sluży zbiór testowy, zwany
też walidacyjnym. Celem rozdzielenia tego zbioru od danych testowych jest weryfikacja
sieci na danych, które nie zostały przez nią przetworzone podczas treningu.

\section{Budowa sieci neuronowej}
\paragraph{Sieć neuronowa} \mbox{}\\
Sieć neuronowa \textit{(ang. ANN Artificial Neural Network)} to struktura matematyczna,
składająca się z neuronów połączonych w warstwy, mająca odzwierciedlać działanie
biologicznych sieci neuronowych, a w szczególności mózgu \cite{intuitiveExplanation, WIKIcnn}.
Sieci mają szerokie zastosowanie w bardzo wielu dziedzinach, co przejawia się ich
popularyzacją w ostatnich latach. Wymagają kosztownego obliczeniowo i czasowo
procesu uczenia, podczas którego dostosowują się do danego problemu. Użycie wytrenowanej
sieci, nie wymaga powtarzania uczenia, co pozwala na jej natychmiastowe wykorzystanie.\\
W dziedzinie rozpoznawania obrazów sieci są w stanie zidentyfikować elementy na obrazach,
bez wcześniejszej znajomości lub wiedzy na temat przedmiotu który mają rozpoznać.

\paragraph{Neuron} \mbox{}\\
Neuron jest najmniejszym elementem sieci neuronowej, posiadającym wiele wejść i jedno wyjście \cite{CS231n_activ, NNbiology, NeuronAnimation}.
Z neuronów w poprzednich warstwach, na każde z wejść docieraja sygnały, które mają 
przypisaną wagę. W neuronie obliczana jest suma ważona wejść, od której odejmowana jest
wartość progowa. Jeśli suma przekroczy wartość progową, neuron jest uaktywniany, a suma
przekazywana jest jako argument funkcji aktywacji neuronu na jego wyjście.\\
Poniższy wzór \ref{eq:neuron} przedstawia sumę ważoną wejść neuronu wraz z dodatkową wagą.
\begin{figure}[h!]
\renewcommand{\figurename}{Wzór}%
\begin{equation} \label{eq:neuron}
f(x_{i}) = f\Big(\sum_{i}w_{i}x_{i} + b \Big)\\
\end{equation}
\caption{Suma ważona w neuronie wraz z dodatkową wagą \textit{b}}
\end{figure}

\paragraph{Wagi neuronu} \mbox{}\\
Jak opisano już wcześniej, neurony połączone są między sobą dzięki licznym wejściom.
Każde takie połączenie ma przypisaną wagę, która zmienia się podczas treningu,
dostosowując sieć neuronową do danych treningowych. Efektem tego jest osiąganie
coraz lepszych wyników w miarę trwania procesu uczenia.

\paragraph{Bias} \mbox{}\\
Bias, użyty już we wzorze \ref{eq:neuron} na sumę wag w neuronie jest dodatkową wagą wejściową,
umożliwiająca jego lepsze dopasowanie neuronu do danych treningowych \cite{needForBias}.

\paragraph{Warstwa} \mbox{}\\
Neurony w sieci zorganizowana są w warstwach. Komunikacja odbywa się tylko między kolejnymi
warstwami, neurony w danej warstwie nie są ze sobą połączone \cite{CS231n, substBigConv}.
Istnieje wiele rodzajów warstw, a sygnał przechodzący przez całą sieć zaczyna się w tzw.
warstwie wejściowej oraz kończy w tzw. warstwie wyjściowej.

\paragraph{Funkcja kosztu} \mbox{}\\
Do porównywania wyników otrzymywanych przez sieć neuronową wykorzystywana jest funkcja kosztu,
funkcja oceniajaca lub funkcja błędu. Jest ona niezbędna do prawidłowego przeprowadzenie
procesu uczenia. Funkcja dostarcza informacje o różnicy między obecnym
stanem sieci, a optymalnym rozwiązaniem dla danych treningowych. Algorytm uczenia
analizuje wartość funkcji kosztu w kolejnych krokach w celu jej zminimalizowania.\\
Najczęściej wykorzystywaną funkcją kosztu jest błąd średniokwadratowy \ref{eq:mse}.
W tej pracy z uwagi na posiadanie 6 możliwych do uzyskania wyników na wyjściu sieci,
zastosowano wielowymiarową entropię krzyżową (\textit{ang. Categorical cross-entropy}) \ref{eq:categorical-crossentropy},
która jest preferowana ponieważ uwypukla aż tak bardzo nieprawidłowych wartości \cite{whyNotMSE}. Jest również
sugerowana przez bibliotekę Keras oraz wykorzystywana w wielu przykładowych modelach sieci neuronowych.
\begin{figure}[h!]
\renewcommand{\figurename}{Wzór}%
\begin{equation} \label{eq:mse}
J(\eta) = \frac {1}{2} \sum_{i}^{m} \Big(h_{\theta}(x^{(i)}) - y^{(i)} \Big)^2\\
\end{equation}
\caption{Błąd średniokwadratowy}
\end{figure}
\begin{figure}[h!]
\renewcommand{\figurename}{Wzór}%
\begin{equation} \label{eq:categorical-crossentropy}
L_{i} = - \sum_{j} t_{i,j} log(p_{i,j})\\
\end{equation}
\centering
gdzie: p - predykcje, t - cele, i - wartość, j - klasa
\caption{Wielowymiarowa entropia krzyżowa}
\end{figure}

\paragraph{Współczynnik uczenia} \mbox{}\\
Aby dostosować stopień w jakim sieć neuronowa będzie dostosowywać się do danych wykorzystujemy
współczynnik lub wskaźnik uczenia się \textit{(ang. lerning rate)}. Często jest
uzależniony od wyboru rodzaju optymalizatora. Wybranie zbyt małego współczynnika wydłuży
proces uczenia, a w skrajnych przypadkach sieć nie zdąży dotrzeć do minimum.
Jeśli zostanie wybrany za duży współczynnik, może spowodować problemy ze znalezieniem
optymalnego rozwiązania. W niektórych algorytmach współczynnik uczenia zmniejsza się w czasie,
by lepiej dostosować się do danych i zniwelować oba wspomniane problemy.

\section{Propagacja wsteczna}
Propagacja wsteczna lub wsteczna propagacja błędów \textit{(ang. Backpropagation)}
jest algorytmem uczenia sieci neuronowych \cite{CS231n_backprop, backprop}.
Służy do wyliczenia gradientu funkcji kosztu, który informuje o szybkości spadku wartości tej funkcji
w danym kierunku z uwzględnieniem wag neuronów oraz biasu. Obliczenie gradientu w sieci propagowane
jest od warstwy wyjściowej do wejściowe, czemu algorytm zawdzięcza swoją nazwę.

\section{Konwolucyjna sieć neuronowa}

Przy wykorzystywaniu sieci neuronowych do operowania na zdjęciach, pojawia się problem
dużą ilością parametrów odpowiadających wartościom każdego z pikseli.
W celu ich zmniejszenia, stosuje się konwolucyjne sieci neuronowe \textit{(ang. CNN - Convolutuinal Neural Network)}.
Do kolejnych warstw zamiast przekazywania informacje o wszystkich pikselach znajdujących się
na obrazach, sieć analizuje obraz przy użyciu filtrów konwolucyjnych i
przesyła dalej informacje o zaobserwowanych cechach.\\
Obecnie ten typ sieci odnosi największe osiągnięcia w dziedzinie rozpoznawania obrazów,
w wielu przypadkach dorównując lub nawet pokonująć ludzkie wyniki.

\paragraph{Konwolucja} \mbox{}\\
Wymieniona wyżej konwolucja, inaczej splot polega na złożeniu dwóch funkcji. W przypadku obrazów
jedna z tych funkcji to obraz, który ma rozmiary większe niż druga funkcja określana
mianem filtra konwolucyjnego. Zastosowanie splotu, w zależności od przypadku,
pozwala na rozmycie, wyostrzanie lub wydobycie głębi z danego obrazu \cite{konwolucja}.
Poniższy wzór \ref{eq:conv} przedstawia sposób obliczenia splotu dwóch funkcji.
\begin{figure}[h!]
\renewcommand{\figurename}{Wzór}%
\begin{equation} \label{eq:conv}
h[m, n] = (f * g)[m, n] = \sum_j^m \sum_k^n f[j, k] * g[m-j, n-k] \\
\end{equation}
\centering
\captionsetup{justification=centering,margin=1cm}
\caption{Splot funkcji. Funkcja f to dwuwymiarowa macierz z pikselami obrazu. Funkcja g to filtr. Funkcja h to nowootrzymany obraz.}
\end{figure}

\section{Warstwy sieci neuronowej}

W sieciach neuronowych wyróżniamy kilka rodzajów warstw, które poza wejściową i wyjściową,
można dopierać zależnie od sposobu rozwiązania danego problemu i typu posidanych danych.

\paragraph{Wejściowa} \mbox{}\\
W sytuacji gdy danymi jest zbiór obrazów, warstwa wejściowa ma rozmiar identyczny z
wymiarami obrazu. Pierwsza warstwa charakteryzuje się brakiem wejść oraz biasu.

\paragraph{Wyjściowa} \mbox{}\\
Rozmiar warstwy wyjściowej odpowiada ilości klas do jakiej dane były klasyfikowne.
W tej pracy, gdzie czekiwanym wyjściem była liczba oczek możliwych do wyrzucenia na
kostce, odpowiada to 6 klasom. Wyjściem takiej sieci jest więc wektor o wymiarach 6x1.

\paragraph{Konwolucyjna} \mbox{}\\
Warstwa konwolucyjna służy do przetworzenia danych z poprzedniej warstwy przy użyciu
filtrów konwolucyjnych \cite{CS231n}. Filtry mają określone wymiary i służą do znajdowania cech
na obrazach lub ich fragmentach. Zastosowanie wielu warstw konwolucyjnych umożliwia filtrom
analizowanie bardziej złożonych zależności na obrazach.

\paragraph{W pełni połączona} \mbox{}\\
Najbardziej rozpowszechnionymi typami warstw są w pełni połączone \textit{(ang. Fully Connected, Dense)}.
Każdy neuron łączy się ze wszystkimi neuronami następnej warstwy. Skutkuje to dużą ilością
potrzebnych do wykonania obliczeń, choć są stosunkowo proste do zaapikowania.
W sieciach konwolucyjnych umieszczane są po lub między warstwami konwolucyjnymi
i służą do powiązania nieliniowych kombinacji, które zostały wygenerowane przez
warstwy konwolucyjne oraz ich sklasyfikowania.

\paragraph{Flatten} \mbox{}\\
Warstwa spłaszczająca \textit{(ang. Flatten)} stosowana jest w celu połączenia warstw
konwolucyjnych z warstwami w pełni połączonymi. Realizowane jest to poprzez przekształcenie
warstwy wejściowej do jednowymiarowego wektora, który następnie służy za wejście
do kolejnych warstw.

\paragraph{Odrzucająca} \mbox{}\\
Użycie warstwy odrzucającej \textit{(ang. Dropout)} jest jednym z najlepszych sposobów
na poradzenie sobie ze zjawiskiem przetrenowania, opisanym na końcu tego rozdziału
\cite{DropoutPreventOverfit}. Warstwa odrzucająca nie wykorzystuje wyjść pewnych neuronów,
co zapobiega rozległym zależnościom między neuronami. W warstwie tej określamy prawdopodobieństwo
z jakim neurony zostaną zachowane. Najczęściej stosuje się ją po warstwach w pełni
połączonych, zarówno przy przechodzeniu w przód jak i tył \cite{DropConnect}.

\paragraph{Pooling} \mbox{}\\
Warstwa ta wykorzystywana jest do zmniejszenia rozmiaru pamięci oraz ilości obliczeń
w sieci neuronowej. Operacja polega na wybraniu jednego piksela z danego obszaru i przekazaniu
go do następnej warstwy. Najczęściej wykorzystywaną warstwą poolingową jest MaxPooling, wybierający
piksel o największej wartości. Pooling jest krytykowany za brak zachowywania informacji
o położeniu piksela przekazanego na wyjście warstwy, co może objawiać się błędnymi
interpretacjami danych przez sieci.

\section{Funkcje aktywacji}

Jeśli wartość progowa danego neuronu zostanie przekroczona, to suma wag zostaje przekazana
jako argument do funkcji aktywacji, a obliczona wartości staje się wyjściem neuronu.
Wybór funkcji jest jednym z kluczowych parametrów danej sieci, ponieważ niewłaściwy wybór
może prowadzić do problemów podczas uczenia sieci.

\paragraph{Sigmoid} \mbox{}\\
Sigmoid popularną funkcją aktywacji \ref{eq:sigmoid}. Problemem z nią związanym
jest ryzyko zaniknięcia gradientu, co może prowadzić do problemu tzw. umierającego
neuronu \cite{activationFunctions}. Wystepuje ono, gdy dla danej funkcji aktywacji,
gradient staję się bardzo mały, co jest równoznaczne z zaprzestaniem procesu uczenia.
W sigmoid gradient może zanikać obustronnie.
\begin{figure}[h!]
\renewcommand{\figurename}{Wzór}%
\begin{equation} \label{eq:sigmoid}
f(x) = \sigma(x) = \frac{1}{1 + e^{-x}} \\
\end{equation}
\caption{Funkcja sigmoidalna}
\end{figure}

\paragraph{Tangens hiperboliczny} \mbox{}\\
Tangens hiperboliczny lub tanh \ref{eq:tanh} jest w istocie przekształconą funkcją simgodalną
\cite{activationFunctions, activationFunctionsV2}. Wykorzystanie
jej powoduje większe wahania gradientu.
\begin{figure}[h!]
\renewcommand{\figurename}{Wzór}%
\begin{equation} \label{eq:tanh}
f(x) = tanh(x) = \frac{(e^x - e^{-x})}{(e^x + e^{-x})} = \frac{2}{1+e^{-2x}} - 1 = 2 sigmoid(2x) - 1\\
\end{equation}
\caption{Tangens hiperboliczny}
\end{figure}

\paragraph{ReLU} \mbox{}\\
ReLU \textit{(and. Rectified linear unit)} \ref{eq:relu} jest najpopularniejszą funkcją
aktywacji wykorzystywaną w sieciach neuronowych \cite{CS231n_activ, WIKIrectifier}.
Zasługą tego jest szybki czas uczenia sieci bez znaczącego kosztu w postaci generalizacji
dokładności. Problem z zanikającym gradientem jest mniejszy niż w przypadku funkcji
sigmoidalnej, ponieważ występuje on tylko z jednej strony.
\begin{figure}[h!]
\renewcommand{\figurename}{Wzór}%
\begin{equation} \label{eq:relu}
f(x) = max(0, x)
\end{equation}
\caption{ReLU - Rectified linear unit}
\end{figure}

\paragraph{LeakyReLU} \mbox{}\\
LeakyReLU jest ulepszeniem funkcji ReLU \cite{CS231n_activ} dzięki zastosowaniu niewielkiego
gradientu w sytuacji, dla której ReLU jest nieaktywne. Zmiana ta pozwala na uniknięcie
problemu znikającego gradientu. Jej wzór \ref{eq:leakyrelu} przedstawiony jest poniżej.
\begin{figure}[h!]
\renewcommand{\figurename}{Wzór}%
\begin{equation} \label{eq:leakyrelu}
f(x) =
\begin{cases}
x & \text{if } x \geqslant 0 \\
0.01x & \text{if } x < 0 \\
\end{cases}
\end{equation}
\caption{LeakyReLU}
\end{figure}

\paragraph{Softmax} \mbox{}\\
Softmax jest funkcją obliczającą rozkład prawdopodobieństwa wystąpienia danego zdarzenia
ze wszystkich możliwych. Wykorzystywana jest w zadagnieniach wymagających przyporządkowania
elementu do jednej z wielu do wielu klas. Wzór funkcji \ref{eq:softmax} zaprezentowany jest poniżej.
\begin{figure}[h!]
\renewcommand{\figurename}{Wzór}%
\begin{equation} \label{eq:softmax}
f(x_{i}) = \frac{e^{x_{i}}} {\sum_{j = 0}^{k}(e^{x_{j}}}
\end{equation}
\caption{Funkcja softmax}
\end{figure}

\section{Optymalizator}

Optymalizator to określenie algorytmu optymalizacyjnego wykorzystywanego do obliczania wag neuronów
i biasu w sieci neuronowej. Posiada kluczowe znaczenie podczas procesu uczenia sieci,
zarówno w kwestii czasu oraz skuteczności. Z tego powodu to zagadnienie jest obiektem
wielu obecnych badań i rozwoju sieci neuronowych \cite{typesOfOptimizationAlgorithms}.

\paragraph{Metoda największego spadku} \mbox{}\\
Metoda największego spadku \textit{(ang. Gradient Descent)} jest podstawowym algorytmem
służacym do zmiany wartości wag oraz biasu podczas procesu uczenia sieci \ref{eq:gradientdescent}.
Wadą tej metody jest przeprowadzanie jednorazowej aktualizacji po wyliczeniu gradientu dla
całego zestawu danych. Spowalnia to uczenie i może powodować problem z ilością zajmowanego
miejsca w pamięci. Największą wadą jest możliwość doprowadzenia do stagnacji w jednym z
lokalnych minimów funkcji.
\begin{figure}[h!]
\renewcommand{\figurename}{Wzór}%
\begin{equation} \label{eq:gradientdescent}
\theta = \theta - \eta * \nabla J(\theta)
\end{equation}
\caption{Metoda największego spadku}
\end{figure}

\paragraph{Stochastic Gradient Descent} \mbox{}\\
Stochastic Gradient Descent \textit{(skrót. GDA)} jest rozwinięciem metody gradientu
prostego, bardzo często wykorzystywana w praktyce \textit{Opis SGD} \cite{OptimizersOverview}.
Ulepszenie polega na obliczaniu gradientu dla jednego lub niewielkiej ilości przykładów
treningowych. Najczęściej korzysta się z więcej niż jednego przykładu, co zapewnia
lepszą stabilność oraz wykorzystuje zrównoleglenie obliczeń. SGD zapewnia większą rozbieżność
niż metoda gradientu prostego, co umożliwia znajdowanie nowych lokalnych minimów, ale
wiąże się z koniecznością zastosowania mniejszego współczynnika uczenia.
\begin{equation}
\theta = \theta - \eta * \nabla J(\theta; x_i; y_i)
\end{equation}

\paragraph{RMSprop} \mbox{}\\
RMSprop umożliwia obliczanie gradientu dla każdego parametru z osobna i zapobiega
zmniejszaniu się współczynnika uczenia \textit{Opis RMSprop} \cite{RMSpropOptimizer, OptimizersOverview}.
Algorytm dostosowuje współczynnik uczenia dla każdej wagi,
bazując na wielkości jej gradientu.

\paragraph{Adam} \mbox{}\\
Adam to skrót od angielskiej nazwy \textit{Adaptive Moment Estimation} i jest rozwinięciem
metody Stochastic Gradient Descent \textit{Opis metody Adam} \cite{AdamOptimizer, OptimizersOverview}.
Metoda ta pozwala na obliczanie z osobna gradientu dla
każdego parametru oraz każdej zmiany momentum. Zapobiega dodatkowo zmniejszającemu się
wskaźnikowi uczenia, a co najważniejsze jest bardzo szybka i pozwala na sprawne uczenie
się sieci. W tej metodzie oblicza się dwa momenty \textit{m} oraz \textit{v}.
\begin{equation}
\begin{align*}
\hat{m_t} = \frac{m_t} {1 - \beta^t_1}, \\
\hat{v_t} = \frac{v_t} {1 - \beta^t_2}, \\
\end{align*}
\end{equation}
Obliczone momenty podstawiane są do wzoru
\begin{equation}
\theta_{t+1} = \theta_t - \frac {\eta} {\sqrt{\hat{v_t}} + \epsilon} \hat{m_t}
\end{equation}
gdzie najczęściej \textbeta \textsuperscript{t}\textsubscript{1} = 0.9 oraz
\textbeta \textsuperscript{t}\textsubscript{2} = 0.99 a \straightepsilon = 10^{-8}\\

\section{Procesy}

\paragraph{Uczenie} \mbox{}\\
Proces uczenia bądź treningu sieci służy zmianie wartości wag, najczęściej zainicjowanych
pseudolosowymi wartościami oraz biasu. Uczenie sieci neuronowej jest bardzo
kosztowne obliczeniowo, co wręcz uniemożliwiało trenowanie modeli w przeszłości,
a obecnie jest jednym z powodów dużego zainteresowania rozwojem technologicznym kart
graficznych. Operacje dodawania oraz mnożenia wektorów i macierzy wykonywane są miliony razy,
mogą być przyśpieszone dzięki możliwościom zrównoleglenia obliczeń.\\

\paragraph{Epoka} \mbox{}\\
Proces uczenia sieci podzielony jest na epoki. Każda epoka odpowiada przejściu
wszystkich elementów z treningowego zbioru danych przez sieć. Ilość epok, podczas których sieć będzie się uczyć ustala się na co najmniej kilkanaście. W przypadku
większych zbiorów danych lub większych modeli ilość epok jest zwiększana.\\
Często spotykaną praktyką w wielu pracach naukowych jest przedstawianie wyników
dla sieci po 100 epokach treningu.

\paragraph{Testowanie} \mbox{}\\
Model sieci neuronowej poddawany ocenie, dzięki której można określić, w jakim stopniu
prawidłowo rozpoznaje obrazy. W przypadku wytrenowanych modeli istotne jest, aby zbiór służący do
testowania nie był wcześniej użyty do treningu sieci. Nauczony model powinien być w stanie
rozpoznawać nowe, nieużyte podczas procesu uczenia dane i poprawnie je klasyfikować.

\paragraph{Predykcja} \mbox{}\\
Wartości zwrócone przez sieć po umieszczeniu w niej określonych danych są określane
mianem predykcji. Pozwala to na wykorzystanie nauczonego modelu w praktycznym
zastosowaniu.

\paragraph{Przetrenowanie} \mbox{}\\
Zjawisko przetrenowania \textit{(ang. Overfitting)}, wspomniane przy opisywaniu warstwy
odrzucającej, jest jednym z największych problemów podczas pracy z sieciami neuronowymi.
Podczas trenowania sieci, możemy obserwować oczekiwany wzrost dokładności i spadek wartości
funkcji kosztu. Przetrenowanie wystepuje kiedy dla testowych danych sprawność sieci
drastycznie spada. Istnieje duże ryzyko, że autor sieci nie będzie świadomy występowania tego
problemu, który w ostateczności może doprowadzić do konieczności tworzenia sieci od nowa.
sieci \cite{DropoutPreventOverfit}.
Proces ten polega na nie wykorzystywaniu wyjść pewnych neuronów, zarówno
w przypadku przechodzenia w przód oraz w tył \cite{DropConnect}. Stosuje się ją po warstwach w pełni
połączonych, w celu zapobiegania rozległym zależnościom między neuronami. W warstwie
tej określone jest prawdopodobieństwo \textit{p} z jakim neuron zostanie zachowany
w warstwie oraz \textit{p - 1} z jakim zostanie odrzucony. Najczęstsza wartość jest
z zakresu 0,5-0,8.

\section{Inne}

\paragraph{ILSVRC} \mbox{}\\
ImageNet Large Scale Visual Recognition Competition \cite{ILSVRC} to coroczny konkurs organizowany
od 2010 roku, w którym naukowcy walczą o najlepszy wynik w dziedzinie rozpoznawania
obrazów przez skonstruowane przez siebie algorytmy. Sieć AlexNet, na której bazowało
kilka stworzonych sieci neuronowych, została stworzona na potrzeby tego konkursu
w 2012 roku.

\paragraph{MNIST} \mbox{}\\
Baza danych MNIST \cite{MNIST} to zbiór 60000 treningowych i 10000 testowych czarno-białych obrazów
w rozmiarze 28x28x1, zawierających ręcznie napisane cyfry 0-9. Jest jednym z
najpopularniejszych zbiorów służących do rozpoczęcia nauki sztucznych sieci neuronowych
i uczenia maszynowego.

\paragraph{CIFAR-10} \mbox{}\\
Zbiór CIFAR-10 \cite{CIFAR-10} składa się z 50000 treningowych i 10000 testowych kolorowych obrazów w rozmiarze
32x32x2. Jest podzielony na 10 klas: samolot, samochód, ptak, kot, jeleń, pies, żaba,
koń, statek, ciężarówka; gdzie każdej z nich przypada po 6000 obrazów. Jest to podobnie
jak MNIST jeden z najbardziej popularnych zbiorów do uczenia maszynowego i sieci neuronowych.

\\\\\\
Tematy do uwzględnienia:
minibatch,
momentum,
one-hot encoding,
