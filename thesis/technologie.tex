% !TEX encoding = UTF-8 Unicode

\chapter{Technologie}
\section{Język programowania i środowisko}

\paragraph{Python} \mbox{}\\
Język programowania Python obecnie jest bardzo popularnym narzędziem wykorzystywanym
w pracy naukowej. Jest to spowodowane bardzo czytelną i zwięzłą składnią,
która w pełni pozwala skupić się na danym problemie. Python jest on także domyślnym
językiem używany w bibliotekach wykorzystywanych do tworzenia modeli sieci neuronowych.

\paragraph{Jupyter Notebook} \mbox{}\\
Aplikacja Jupyter Notebook pozwala uruchamiać w przeglądarce pliki, nazywane notebookami,
które składają się z wielu bloków. W blokach może znajdować się wykonywalny kod programu
lub jego fragment, a w pozostałych można prezentować m.in. teksty, wykresy bądź tabele.
Korzystanie z Jupyter Notebooka zdecydowanie ułatwia pracę, umożliwiając tworzenie
kodu wraz z podglądem wykresów bądź danych prezentowanych w innej formie, bez
konieczności przełączania między oknami bądź kartami danego środowiska programistycznego.\\
W tej pracy wykorzystywana jest to uruchamiania programów w języku Python.

\section{Biblioteki}

\paragraph{OpenCV} \mbox{}\\
Biblioteka funkcji do obróbki obrazów, najczęsciej wykorzystywana w językach C++
oraz Python. W projekcie została użyta do uzyskiwania zbiorów obrazów kości do gry
ze zdjęć wykonanych kamerą.

\paragraph{TensorFlow} \mbox{}\\
Biblioteka do uczenia maszynowego oraz tworzenia sieci neuronowych. Jej ogromną
zaletą jest implementacja w języku C++ umożliwiająca wykorzystywanie procesorów oraz
kart graficznych. Z uwagi na charakter wykonywanych obliczeń praca przy użyciu
kart graficznych jest kilkukrotnie szybsza niż na procesorze, co znacząco przyśpiesza
proces uczenia. Biblioteka najlepiej wspiera języki programowania do C++ oraz Python.

\paragraph{Keras} \mbox{}\\
Biblioteka do tworzenia modeli sieci neuronowych, wykorzystująca bardziej niskopoziomowe
biblioteki jak TensorFlow, Theano lub CNTK. Zaletami Kerasa są
zarówno przystępny interfejs pozwalający w krótkim czasie stworzyć model sieci
oraz możliwość tworzenia zaawansowanych modeli przy średnim pogorszeniu czasu
uczenia się sieci o 3-4\% w stosunku do bibliotek, na których bazuje.\\
W sytuacji kiedy interfejs oraz dokumentacja do TensorFlow mogą być dla początkującej
osoby niezrozumiałe, jest to świetna alternatywa do wdrożenia się w to zagadnienie.

\paragraph{Numpy} \mbox{}\\
Moduł języka Python umożliwiający wykonywanie zaawansowanych operacji na macierzach
oraz wektorach, wspierający liczne funkcje matematyczne. Jest bardzo rozpowszechniony
i wykorzystywany w wielu, głównie naukowych zastosowaniach. Numpy wprowadza własne
typy danych oraz funkcje niedostępne w standardowej instalacji Pythona. Może być
rozbudowany o moduł Scipy, który nie był jednak wykorzystany przy tworzeniu tej pracy.

\paragraph{Matplotlib} \mbox{}\\
Narzędzie do tworzenia wykresów dla języka Python oraz modułu Numpy. Z uwagi na
bardzo duże możliwości jest bardzo popularny, pozostając jednocześnie prostym w
użyciu. Zawiera moduł pyplot, który w założeniu ma maksymalnie przypominać interfejs
w programie MATLAB.

\paragraph{LaTeX} \mbox{}\\
Oprogramowanie do składu tekstu, nie będące edytorem tekstowy typu WYSIWYG.
LaTeX bazuje na TeX, który jest systemem składu drukarskiego do prezentacji w
formie graficznej. Ogromną zaletą jest możliwość tworzenia w tekście zaawansowanych
wzorów matematycznych. Tekst niniejszej pracy napisany został przy pomocy tego narzędzia.

\section{Technologie wspomagające}

\paragraph{NVIDIA CUDA} \mbox{}\\
Równoległa architektura obliczeniowa firmy NVIDIA pozwala na wielokrotne
przyśpieszneie obliczeń podczas uczenia się sieci neuronowych. Dzięki bibliotekom
takim jak TensorFlow lub Keras, które wspierają obliczenia na kartach graficznych
czas precesu uczenia drastycznie maleje. Podczas tej pracy wykorzystana została
karta NVIDIA TESLA K80 12GB GPU dostępna na Amazon AWS oraz Google Compute Engine.

\paragraph{Amazon AWS EC2} \mbox{}\\
Platforma z wirtualnymi maszynami zwanymi instancjami, które można dostosować
zależnie od potrzeb klienta. Usługa działa na zasadzie rozliczenia godzinowego
podczas korzystania z niej. W tej pracy zostały wykorzystane instancje
zoptymalizowane do obliczeń na kartach graficznych i uczenia maszynowego,
wyposażone we wcześniej wspomnianą kartę NVIDIA TESLA K80. Warto
zwrócić uwagę, że na obu platformach czas uczenia sieci zmniejszył się średnio 6-10
krotnie w stosunku do pracy na komputerze wykorzystującym procesor.
