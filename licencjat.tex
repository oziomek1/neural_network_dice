% !TEX encoding = UTF-8 Unicode

\documentclass[a4paper,oneside]{report}

\usepackage{titlesec}
\usepackage[top=0cm, bottom=0cm, left=2cm, right=2cm]{geometry}
\usepackage[T1]{fontenc}
\usepackage[polish]{babel}
\usepackage[utf8]{inputenc}
\usepackage{lmodern}
\selectlanguage{polish}
\makeatletter

\renewcommand{\maketitle}{\begin{titlepage}
	
	\vspace*{ \stretch{1} }
	\begin{center} \LARGE
	Uniwersytet Jagielloński \\
  	Wydział Fizyki, Astronomii i Informatyki Stosowanej
	\end{center}
	
	\vspace{ \stretch{1} }
	\begin{center} \Huge 
		\textsc {\@title}
	\end{center}
			
	\vspace{ \stretch{1} }
	\begin{center}
		\Large \@author \\
		\vspace{3mm}
		\large Nr albumu: 1124802
	\end{center}
	
	\vspace{ \stretch{1.5} }
	\begin{flushright}
	\begin{minipage}{8cm}
	\begin{center} \Large
	Praca wykonana pod kierunkiem \\
	prof. dr hab. Piotra Białasa \\
	kierownika Zakładu Technologii Gier \\
	FAIS UJ
	\end{center}
	\end{minipage}
	\end{flushright}
	\vspace{ \stretch{1} }
	\begin{center}
	\@date
	\end{center}
\end{titlepage}
}
\makeatother

\titleformat{\chapter}[display]
  {\normalfont\bfseries}{}{0pt}{\Huge}

\renewcommand*\contentsname{Spis Treści}

\author{Wojciech Ozimek}
\title{Wykorzystanie sztucznych sieci neuronowych do analizy obrazów na przykładzie kostki do gry}
\date{kwiecień 2018}

\begin{document}
\maketitle{}

\tableofcontents
\newpage


\chapter {Wstęp}

\paragraph {Lorem  ipsum  dolor  sit  amet,  consectetuer  adipiscing  
elit.   Etiam  lobortisfacilisis sem.  Nullam nec mi et 
neque pharetra sollicitudin.  Praesent imperdietmi nec ante. 
Donec ullamcorper, felis non sodales...}


\chapter {Technologie}

\section {Język programowania i środowisko}

\subsection {Python}

\subsection {Jupyter Notebook}

\section {Biblioteki}

\subsection {OpenCV}

\subsection {Tensorflow}

\subsection {Keras}

\subsection {Numpy}

\subsection {Matplotlib}

\subsection {LaTeX}

\section {Technologie poza programistyczne}

\subsection{Nvidia CUDA}

\subsection {Amazon AWS EC2}

\subsection {Google Compute Engine}


\chapter {Założenia pracy}


\chapter {Sieć neuronowa}

\section {Czym jest sieć neuronowa}

\section {Konwolucyjna sieć neuronowa}


\end{document}
